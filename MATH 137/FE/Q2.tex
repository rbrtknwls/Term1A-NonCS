\documentclass[11pt]{article}
\textwidth 15cm 
\textheight 21.3cm
\evensidemargin 6mm
\oddsidemargin 6mm
\topmargin -1.1cm
\setlength{\parskip}{1.5ex}


\usepackage{amsfonts,amsmath,amssymb,enumerate}



\begin{document}
\parindent=0pt

\textbf{Q02a} Let b be a real number. We know from the "Critical Point" Theorem (Page 202) that a critical point will happen at some c when:
\[ f'(c) = 0 \text{ or } f'(c) = DNE \]
Thus in order to solve we must first find the first derivative.
\begin{align*}
f(x) & = x^{\frac{1}{3}} + bx^{\frac{4}{3}}\\
\frac{dy}{dx}(f(x)) & = \frac{dy}{dx}(x^{\frac{1}{3}} + bx^{\frac{4}{3}})\\
f'(x) & = \frac{dy}{dx}(x^{\frac{1}{3}}) + \frac{dy}{dx}(bx^{\frac{4}{3}})\\
f'(x) & = \frac{1}{3}x^{(\frac{1}{3}-1)} + b\frac{4}{3}x^{(\frac{4}{3} -1)}\\
f'(x) & = \frac{1}{3x^{\frac{2}{3}}} + b\frac{4}{3}x^{\frac{1}{3}}
\end{align*}
Since we can see an $x$ is in the denominator for the first term, this would imply that f'(x) is undefined at 0 or:
\[ f'(0) = DNE \]
Since there is only one possible value such that$ f'(c) = DNE$, we will now try to solve $f'(c) = 0$, starting by setting f'(x) to be zero:
\begin{align*}
0 & = \frac{1}{3x^{\frac{2}{3}}} + b\frac{4}{3}x^{\frac{1}{3}} \\
 -\frac{1}{3x^{\frac{2}{3}}} & = b\frac{4}{3}x^{\frac{1}{3}} \\
 -\frac{1}{3x^{\frac{2}{3}}}x^{\frac{2}{3}} & = b\frac{4}{3}x^{\frac{1}{3}}x^{\frac{2}{3}} \\
 -\frac{1}{3} & = b\frac{4}{3}x \\
 -\frac{1}{4b} & = x
\end{align*}
Therefore our possible critical points will be when:
\[ x = 0 \text{ or }  -\frac{1}{4b} \] \\\\\\\\\\\\\\

\textbf{Q02b} We know we have two possible values to test if a local minimum can exist, 0 and $-\frac{1}{4b}$ (for some real number b). This is because minimum values points can only exist at critical points\\\\
We know that f(x) is continuous as if you replace $x$ with the $z = x^1/3$ (which is continuous), you get:
\[ z + bz^4 \]
Which is also continuous as its a polynomia,l thus $f(x)$ is a continuous. Therefore in order to see if 0 in a minimum we shall use "The First Derivative Test" (Page 223),  we will use the interval $(0^-,0^+)$ to check if:
\begin{align*}
f'(0^-)<f'(x) < 0 &\text{ for all $x \in (0^-,0)$} \\
f'(0^+)>f'(x) > 0 &\text{ for all $x \in (0,0^+)$}
\end{align*}
Evaluating the first equation we find a contradiction at $f'(0^-)$:
\begin{align*}
f'(0^-)< f'(x) < 0 \\
\frac{1}{3(0^-)^{\frac{2}{3}}} + b\frac{4}{3}(0^-)^{\frac{1}{3}}< f'(x) < 0 \\
\frac{1}{3(0^+)} + b\frac{4}{3}(0^-)< f'(x) < 0 \\
\frac{1}{(0^+)} + (0^-)< f'(x) < 0 \\
\infty + (0^-)< f'(x) < 0 \\
\infty < f'(x) < 0 
\end{align*}
Because this contradicts the First Derivative Test this means that 0 can not be a local minimum.\\
Moving on we can test the critical point $-\frac{1}{4b}$ using the "Second Derivative Test" (we can use this as f'($-\frac{1}{4b}$ ) = 0). But before we do this we must first find the second derivative:
\begin{align*}
f'(x) & = \frac{1}{3}x^{-\frac{2}{3}} + b\frac{4}{3}x^{\frac{1}{3}}\\
\frac{dy}{dx}f'(x) & = \frac{dy}{dx}(\frac{1}{3}x^{-\frac{2}{3}}) + \frac{dy}{dx}(b\frac{4}{3}x^{\frac{1}{3}})\\
f''(x) & = \frac{1}{3}\frac{-2}{3}x^{-\frac{5}{3}} + b\frac{4}{3}\frac{1}{3}x^{-\frac{2}{3}}\\
f''(x) & = \frac{-2}{9}x^{-\frac{5}{3}} + b\frac{4}{9}x^{-\frac{2}{3}}
\end{align*}
Now we will plug in the critical point (c) and simplify:
\begin{align*}
f''(c) & = \frac{-2}{9}(-\frac{1}{4b})^{-\frac{5}{3}} + b\frac{4}{9}(-\frac{1}{4b})^{-\frac{2}{3}}\\
f''(c) & = \frac{2}{9}(-\frac{1}{4b})^{-\frac{2}{3}}(-(-\frac{1}{4b})^{-1}+2b) \\
f''(c) & = \frac{2}{9}(-4b)^{\frac{2}{3}}(-(-4b)^{1}+2b) \\
f''(c) & = \frac{2}{9}\sqrt[\leftroot{-2}\uproot{2}3]{(16b^2)}(6b) 
\end{align*}
The "Second Derivative Test" tells us that a minimum will exist at the critical point if:
\[ f''(c) > 0 \]
As we can see from the above equation, the positivity is of the equation is solely reliant on the $6b$ term as ($b^2$ will always be positive). Therefore $f'(c) > 0$ if b is also positive. In other words the minimum will exist as along as:
\[ b > 0 \]
Therefore we have shown that the only possible local minimum exists when b is positive and the minimum will exist at:
\[ x= -\frac{1}{4b} \]
\\\\\
\textbf{Q02c} We know that intervals (of increase and decrease) will only change sign values critical points, therefore our possible intervals are :
\begin{center}
 \begin{tabular}{||c c c c c||} 
 \hline
 ($-\infty$, $-\frac{1}{4b})$ & $-\frac{1}{4b}$ & $(-\frac{1}{4b}, 0)$ & 0 & $(0,\infty)$  \\ [0.5ex] 
 \hline\hline
& 0 & + & Undefined &\\ 
 \hline
\end{tabular}
\end{center}
Since $b > 0$ the minimum will exist, and will have neutral interval (not increasing or decreasing). We also know from the previous question that the rate of change is undefined at 0. To start solving for the rest of the values we will find the first derivative at negative infinity:
\[ f'(-\infty) =  \frac{1}{3x^{\frac{2}{3}}} + b\frac{4}{3}x^{\frac{1}{3}} = \frac{1}{3(-\infty)^{\frac{2}{3}}} + b\frac{4}{3}(-\infty)^{\frac{1}{3}} \approx 0 -\infty  \]
Thus the interval ($-\infty$, $-\frac{1}{4b})$ will be negative, moving onto positive infinity we know that:
\[ f'(\infty) =  \frac{1}{3x^{\frac{2}{3}}} + b\frac{4}{3}x^{\frac{1}{3}} = \frac{1}{3(\infty)^{\frac{2}{3}}} + b\frac{4}{3}(\infty)^{\frac{1}{3}} \approx 0 + \infty  \]
Thus the interval $(0,\infty)$ is positive, so we will have the intervals of increase (+) and decrease (-) given by:
\begin{center}
 \begin{tabular}{||c c c c c||} 
 \hline
 ($-\infty$, $-\frac{1}{4b})$ & $-\frac{1}{4b}$ & $(-\frac{1}{4b}, 0)$ & 0 & $(0,\infty)$  \\ [0.5ex] 
 \hline\hline
- & 0 & + & Undefined & +\\ 
 \hline
\end{tabular}
\end{center}
In order to find the points of concavity we must first find the inflection points, which are determined by:
\[ f''(x) = 0 \text{ or } f''(x) = DNE \]
We know from our Part B that our equation for $f''(x)$ is:
\[ f''(x)  = \frac{-2}{9x^{\frac{5}{3}}} + b\frac{4}{9x^{\frac{2}{3}}}\]
Thus we can tell that this will be indeterminate when x = 0, so this means that x=0 is one of the inflection points. Setting f''(x) = 0 we find that:
\begin{align*}
f''(x) & = \frac{-2}{9x^{\frac{5}{3}}} + b\frac{4}{9x^{\frac{2}{3}}} \\
0 & = \frac{-2}{9x^{\frac{5}{3}}} + b\frac{4}{9x^{\frac{2}{3}}} \\
 \frac{-2}{9x^{\frac{5}{3}}}  & = -b\frac{4}{9x^{\frac{2}{3}}} \\
-2  & = -b\cdot4\cdot x \\
x &= \frac{1}{2b} \\
\end{align*}
And thus our last inflection point will be when $x = \frac{1}{2b}$. When the second derivative is positive we know the function will be concave up (and the opposite for concave down), so we can create a table in order to show the concavity:
\begin{center}
 \begin{tabular}{||c c c c c||} 
 \hline
 ($-\infty, 0$) & 0 & $(0, \frac{1}{2b})$ &  $\frac{1}{2b}$ & $( \frac{1}{2b},\infty)$  \\ [0.5ex] 
 \hline\hline
&  undefined &  & 0 &\\ 
 \hline
\end{tabular}
\end{center}
We know that 0 is undefined from what we saw previously and that $\frac{1}{2b}$ is zero, to start solving the rest we will consider negative infinity:
\[ f''(-\infty) = \frac{-2}{9x^{\frac{5}{3}}} + b\frac{4}{9x^{\frac{2}{3}}} = \frac{-2}{9(-\infty)^{\frac{5}{3}}} + b\frac{4}{9(-\infty)^{\frac{2}{3}}} \approx 0^+ + 0^+\]
Thus the interval  ($-\infty, 0$) will be positive. We will use $\frac{1}{2b}^-$ to check the next interval:
\[ f''(\frac{1}{2b}^-) = \frac{-2}{9x^{\frac{5}{3}}} + b\frac{4x}{9x^{\frac{5}{3}}} = \frac{-2+4bx}{9x^{\frac{5}{3}}} =   \frac{-2+4b\frac{1}{2b}^-}{9(\frac{1}{2b}^-)^{\frac{5}{3}}} \approx -2 + 2^-\]
Since the denominator is positive it is redundant, we can see that the numerator is negative 2 plus a number a little less then 2 so the interval $(0, \frac{1}{2b})$ is negative. We will use $\frac{1}{2b}^-$ to check the next interval:
\[ f''(\frac{1}{2b}^+) = \frac{-2}{9x^{\frac{5}{3}}} + b\frac{4x}{9x^{\frac{5}{3}}} = \frac{-2+4bx}{9x^{\frac{5}{3}}} =   \frac{-2+4b\frac{1}{2b}^+}{9(\frac{1}{2b}^+)^{\frac{5}{3}}} \approx -2 + 2^+\]
Thus the interval $(\frac{1}{2b},\infty)$ will be positive, so the finalized concave up and concave down table will look like:
\begin{center}
 \begin{tabular}{||c c c c c||} 
 \hline
 ($-\infty, 0$) & 0 & $(0, \frac{1}{2b})$ &  $\frac{1}{2b}$ & $( \frac{1}{2b},\infty)$  \\ [0.5ex] 
 \hline\hline
 + &  undefined & -  & 0 & +\\ 
 \hline
\end{tabular}
\end{center}
Combining our interval of increase/decrease table and our concavity table we get:
\begin{center}
 \begin{tabular}{|c |c c c c c c c||} 
 \hline
 &($-\infty$, $-\frac{1}{4b})$ & $-\frac{1}{4b}$ & $(-\frac{1}{4b}, 0)$ &  0 & $(0, \frac{1}{2b})$ &  $\frac{1}{2b}$ & $( \frac{1}{2b},\infty)$  \\ [0.5ex] 
 \hline\hline
 f'(x) &- & 0  & + & undefined & +  & + & +\\ 
 \hline
 f''(x) &+ &  + & + & undefined & - & 0 & +\\ 
\hline
\end{tabular}
\end{center}
So in summery the intervals and inflection points are:
$$
\begin{cases}
\text{Interval of Increase}) \ \  (-\frac{1}{4b}, 0) \cup (0,\infty) \\
\text{Interval of Decrease}) \ \ (-\infty, -\frac{1}{4b}) \\
\text{Interval of Concave Up}) \ \ (-\infty, 0) \cup ( \frac{1}{2b},\infty) \\
\text{Interval of Concave Down}) \ \ (0, \frac{1}{2b}) \\
\text{Inflection Points}) \ \ x = 0,   \frac{1}{2b} \\
\end{cases}
$$
\end{document}