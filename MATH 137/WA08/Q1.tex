\documentclass[11pt]{article}
\textwidth 15cm 
\textheight 21.3cm
\evensidemargin 6mm
\oddsidemargin 6mm
\topmargin -1.1cm
\setlength{\parskip}{1.5ex}


\usepackage{amsfonts,amsmath,amssymb,enumerate}

\begin{document}
\parindent=0pt

\textbf{Robert (Robbie) Knowles MATH 137 Fall 2020: WA08}

\textbf{Q01.} Let $c$ be an arbitrary real number such that ($c > 0$) This means that we start with the equation:
\begin{align*}
f\sqrt{x} +  \sqrt{y} & = \sqrt{c}\\
 \sqrt{y} & = \sqrt{c}-\sqrt{x}\\
y & = (\sqrt{c}-\sqrt{x})^2
\end{align*}
This also impliies that:
\[ f(x) =  (\sqrt{c}-\sqrt{x})^2 \]
If we take the implicit derivative of both sides we will find that:
\begin{align*}
\frac{d}{dx}y & =\frac{d}{dx} (\sqrt{c}-\sqrt{x})^2 \\
\frac{dy}{dx} & =2(\sqrt{c}-\sqrt{x}) \times \frac{d}{dx} [\sqrt{c}-\sqrt{x}]\\
\frac{dy}{dx} & =2(\sqrt{c}-\sqrt{x}) \times [\frac{d}{dx}  \sqrt{c}- \frac{d}{dx} \sqrt{x}]
\end{align*}
Notice that $c$ is a constant and $x$ is of the form $x^{(\frac{1}{2})}$ thus the derivative will become:
\begin{align*}
\frac{dy}{dx} & =2(\sqrt{c}-\sqrt{x}) \times [0 - \frac{1}{2}x^{\frac{-1}{2}}]\\
\frac{dy}{dx} & =2(\sqrt{c}-\sqrt{x}) \times \frac{1}{-2x^\frac{1}{2}}\\
\frac{dy}{dx} & =-\frac{\sqrt{c}-\sqrt{x}}{\sqrt{x}}
\end{align*}
This can also be rewritten as:
\[ f'(x) = -\frac{\sqrt{c}-\sqrt{x}}{\sqrt{x}} \]
Let $x_0$ and $y_0$ be an arbitrary solution to $\sqrt{x} +  \sqrt{y} = \sqrt{c}$, therefore a tangent line will exist that goes through ($x_0$, $y_0$). We can thus apply the Linear Approximation at $x=a$, and we get:
\[ L_a(x) = f(a) + f'(a)(x-a) \]
The next step we will split into two parts, finding the $y$ intercept and finding the $x$ intercept. Starting with the first we know that at the $y$ intercept the value of $x$ will be zero, thus our equation becomes:
\begin{align*}
 L_a(x) & = f(a) + f'(a)(x-a)\\
 L_a(0) & = f(a) + f'(a)(0-a)\\
 & =(\sqrt{c}-\sqrt{a})^2 +f'(a)(-a)\\
 & =(\sqrt{c}-\sqrt{a})^2 - \frac{\sqrt{c}-\sqrt{a}}{\sqrt{a}}(-a)\\
 & =(\sqrt{c}^2 - 2\sqrt{a}\sqrt{c} + \sqrt{a}^2) - (-\sqrt{c}\sqrt{a}+\sqrt{a}\sqrt{a})\\
 & = c - 2\sqrt{a}\sqrt{c} + a +\sqrt{c}\sqrt{a} - a\\
 & = c - \sqrt{a}\sqrt{c}\\
\end{align*}
Our $x$ intercept will happen when $y = 0$, in other terms this is equivilent to $L_a(x) = 0$. This means our equation will become:
\begin{align*}
 L_a(x) & = f(a) + f'(a)(x-a)\\
0 & = f(a) + f'(a)(x-a)\\
 & = f(a) + f'(a)(x-a)\\
& = \frac{f(a)}{f'(a)} + (x - a)\\ 
& = \frac{(\sqrt{c}-\sqrt{a})^2 }{ -\frac{\sqrt{c}-\sqrt{a}}{\sqrt{a}}} + (x - a)\\
& = -\frac{\sqrt{a}(\sqrt{c}-\sqrt{a})^2}{\sqrt{c}-\sqrt{a}} + (x - a) \\
& = -\sqrt{a}(\sqrt{c}-\sqrt{a})) + (x - a) \\
& = -\sqrt{a}\sqrt{c} +\sqrt(a)\sqrt{a} + x - a \\
\sqrt{a}\sqrt{c} & =   x  
\end{align*}
Since $x$ is the $x$ intercept and $L_a(0)$ is the $y$ intercept, the sum of the two will equal:
\begin{align*}
= & L_a(0)  + x\\
= &  \text{ c} - \sqrt{a}\sqrt{c} +\sqrt{a}\sqrt{c} \\
= & \text{ c} 
\end{align*}
Thus showing that the $x$ and $y$ intercepts of the tangent will sum up to $c$.
\end{document}