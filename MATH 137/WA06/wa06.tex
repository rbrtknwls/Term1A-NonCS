\documentclass[12pt]{article}
\textwidth 15cm 
\textheight 21.3cm
\evensidemargin 6mm
\oddsidemargin 6mm
\topmargin -1.1cm
\setlength{\parskip}{1.5ex}


\usepackage{amsfonts,amsmath,amssymb,enumerate}

\begin{document}
\parindent=0pt

\textbf{Robert (Robbie) Knowles MATH 137 Fall 2020: WA06}

\textbf{Q01.} In order for $f(x)$ to be differentiable at $a$

\begin{align*}
f'(a)=\lim_{h \to 0} \frac{f(a + h)-f(a)}{h}
\end{align*}

This must also hold for the left and right limits, such that

\begin{align*}
f'(a)=\lim_{h \to 0^-} \frac{f(a + h)-f(a)}{h} =  \lim_{h \to 0^+} \frac{f(a + h)-f(a)}{h} 
\end{align*}

We will start by finding the value right side limit

\begin{align*}
 \lim_{h \to 0^+} \frac{f(a + h)-f(a)}{h} 
\end{align*}

Notice that as $\lim_{h \to 0^+}$, the value of $a+h > a$, which means that according to the piecewise function $f(a+ h) = (a+h)^2$, also note that $f(a) = a^2$ by definition as well. This means the rightside limit becomes:

\begin{align*}
 \lim_{h \to 0^+} \frac{f(a + h)-f(a)}{h}  &= \lim_{h \to 0^+} \frac{(a + h)^2 -a^2}{h}\\\\
&= \lim_{h \to 0^+} \frac{(a^2+2ah+h^2) - a^2}{h}\\\\
&= \lim_{h \to 0^+} \frac{2ah+h^2}{h}\\\\
&= \lim_{h \to 0^+} 2a+h \\\\
& = 2a
\end{align*}

Now that we have a value for the righthand limit, we will find the values of $a$ and $b$ such that the righthand limit equals the left hand limit, or that:

\begin{align*}
2a = \lim_{h \to 0^-} \frac{f(a + h)-f(a)}{h}
\end{align*}

Notice that  as $\lim_{h \to 0^-}$, the value of $a+h < a$, which means that according to the piecewise function $f(a+ h) = 2(a+h)+b$, and  $f(a) = a^2$  so we get:

\begin{align*}
2a & = \lim_{h \to 0^-} \frac{f(a + h)-f(a)}{h}\\\\
& = \lim_{h \to 0^-} \frac{2(a+h)+b-a^2}{h}\\\\
& = \lim_{h \to 0^-} \frac{2a+2h+b-a^2}{h}\\\\
& = \lim_{h \to 0^-} \frac {2h}{h} +\frac{2a+b-a^2}{h}
\end{align*}

In order to find the limit, the numerator of $\frac{2a+b-a^2}{h}$  must equal zero. This means that we will make $2a+b = a^2$, and that results in:

\begin{align*}
2a & = \lim_{h \to 0^-} \frac {2h}{h} +\frac{2a+b-a^2}{h}\\\\\
& = \lim_{h \to 0^-} 2 +\frac{0}{h} \\\\\
& = 2
\end{align*}

Therefore if $2a = 2$, then $\pmb{a=1}$, if we plug this into $2a+b = a^2$ we get that $2 + b = 1$ or that $\pmb{b=-1}$. In conclusion, since when $\pmb{a=1}$ and $\pmb{b=-1}$ $f'(a$) exists as the left and right derivative of $f$ are equal, it must mean that for those values of $a$ and $b$ that $f(x)$ is differentiable at $a$.
\end{document}