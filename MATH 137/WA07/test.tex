\documentclass[11pt]{article}
\textwidth 15cm 
\textheight 21.3cm
\evensidemargin 6mm
\oddsidemargin 6mm
\topmargin -1.1cm
\setlength{\parskip}{1.5ex}


\usepackage{amsfonts,amsmath,amssymb,enumerate}

\begin{document}
\parindent=0pt

\textbf{Robert (Robbie) Knowles MATH 137 Fall 2020: WA07}\\

\textbf{Q01a.} To start we will assume that for any $x$ that $L_a(x) = L_b(x)$. Therefore we know that for any real number $a,b$ where ($a \neq b$):
\[ L_b(b) = L_a(b)  \text{  and   } L_a(a) = L_b(a)\] 
We then re-write as:
\[ L_b(b) -  L_a(b) = 0  \text{  and   } L_a(a) - L_b(a) = 0\] 
Setting the 0 = 0 we find that:
\[ 0 = 0 \] 
\[ L_b(b) -  L_a(b) =  L_a(a)  - L_b(a) \] 
Re-arranging we find that:
\[L_b(b)+ L_b(a)  = L_a(a) + L_a(b) \]
By the definition of Linear Approximation that we can expand this as:
\begin{align*}
f(b)+f'(b)(b-b)+f(b)+f'(b)(a-b) & =  f(a)+f'(a)(a-a)+f(a)+f'(a)(b-a)\\
f(b)+f(b)+f'(b)(-(b-a)) & = f(a)+f(a)+f'(a)(b-a)\\
2f(b)-f'(b)(b-a)  &= 2f(a)+ f'(a)(b-a)\\
2f(b) - 2f(a) &= f'(b)(b-a) + f'(a)(b-a)\\
2(f(b) - f(a)) &= (f'(b) + f'(a))(b-a)\\
\frac{f(b)-f(a)}{b-a} &= \frac{f'(b)+f'(a)}{2}
\end{align*}
Therefore we have proved that if $L_a(x) = L_b(x)$ the for all $x$ and for any real $a,b$ where ($a \neq b$) that $\frac{f(b)-f(a)}{b-a} = \frac{f'(b)+f'(a)}{2}$\\\\\\\\\\\\\\\\\\\\\\\\\\




\textbf{Q01b.} Note that the question (for $a \neq b$ and all $x$) can be re-written as:
\[ \frac{f(b)-f(a)}{b-a} = \frac{f'(b)+f'(a)}{2}  \implies L_a(x) = L_b(x) \]
In order to disprove this we need to find a counter example where the the hypothesis is true but the conclusion is false. For this case let $f(x)=x^2$ (note that $f'(x) = 2x$) and let $a=3$ and $b=-2$, the hypothesis thus becomes:
\begin{align*}
\frac{f(b)-f(a)}{b-a} &= \frac{f'(b)+f'(a)}{2}\\
\frac{b^2-a^2}{b-a} &= \frac{2b+2a}{2}\\
\frac{(-2)^2-(3^2)}{-2-3} &= \frac{2(-2)+2(3)}{2}\\
\frac{4-9}{-5} &= \frac{2}{2}\\
1 &= 1\\
\end{align*}
Now that we know the hypothesis is correct, we will show that the Linear Approximation of $L_a(x)$ and $L_b(x)$ is equal to:
\[ L_a(x) =f(a)+f'(a)(x-a) = 3^2 + (6)(x-3) = 9 + 6(x-3) \]
\[ L_b(x) =f(b)+f'(b)(x-b) = -2^2 + (-4)(x+2) = 4  - 4(x+2) \]
When $x =1$ notice that:
\[ L_a(1) = 9 + 6(1-3) = 9+6(-2) = -3 \]
\[ L_b(1) = 4  - 4(1+2) = 4 - 4(3) = -8 \]
Therefore at least one $x$ exists such that for a given $a$ and $b$ that satisfies the hypothesis, where $l_a(x) \neq l_b(x)$, thus disproving that $l_a(x) = l_b(x)$ for all real values of x.






\end{document}