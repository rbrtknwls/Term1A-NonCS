\documentclass[11pt]{article}
\textwidth 15cm 
\textheight 21.3cm
\evensidemargin 6mm
\oddsidemargin 6mm
\topmargin -1.1cm
\setlength{\parskip}{1.5ex}


\usepackage{amsfonts,amsmath,amssymb,enumerate}



\begin{document}
\parindent=0pt

\textbf{Q03} To start we know that $z$ (and thus $|z|$) can be expressed as:
\[ z = a + bi \text{ (for a,b $\in \mathbb R$)}  \]
\[ |z| = \sqrt{a^2+b^2}  \]
We thus can plus this equation into the bounds of $|z|$:
\begin{align*}
 1 < & \sqrt{a^2+b^2} < 4  \\
 1 < & a^2+b^2 < 16 
\end{align*}
Let the outer bounds (values a and b must be less then) be in the form (a,b), they can be represented as:
\[ [0,4), [0,-4), (4,0], (-4,0] \]
Let the inner bounds (balues a and b must be greater then) be in the form (a,b), they can be represented as:
\[ [0,1), [0,-1), (1,0], (-1,0] \]
Thus we can see the graph for $|z|$ would be the graph of a circle with radius 4 with a hole in the shape of a circle with radius 1. \\
\[ \text{Shown as the first graph} \]
Adding $i$ would be an upward shift of 1 in the imaginery direction, thus resulting in the second graph where the shaded region of the complex plane represents \mbox{$\left\{i + \overline z \colon 1 < |z| < 4, z \in \mathbb C\right\}$}.

\end{document}