\documentclass[11pt]{article}
\textwidth 15cm 
\textheight 21.3cm
\evensidemargin 6mm
\oddsidemargin 6mm
\topmargin -1.1cm
\setlength{\parskip}{1.5ex}


\usepackage{amsfonts,amsmath,amssymb,enumerate}



\begin{document}
\parindent=0pt

\textbf{Q04A} Let n be a non negative integer. We know from BT we know that we can express $(1+i)^n$ and $(1-i)^n$ as:
\begin{align*}
 (1+i)^n & = \sum\limits_{k = 0}^n\binom{n}{k}(i)^k \\
 (1-i)^n & = \sum\limits_{k = 0}^n\binom{n}{k}(-i)^k \\
\end{align*}
Adding the two together we find that:
\begin{align*}
 (1+i)^n +  (1-i)^n & = \sum\limits_{k = 0}^n\binom{n}{k}(i)^k + \sum\limits_{k = 0}^n\binom{n}{k}(-i)^k\\
 & = \sum\limits_{k = 0}^n\binom{n}{k}[-i^k+i^k] 
\end{align*}
Notice that we can thus split this sigma into two cases, when k is even and when k is odd. Lets start when k is odd, we know that we can express $k$ as $2d$ + 1 (where $d \in \mathbb N$), so our equation becomes:
\begin{align*}
\sum\limits_{k = 0}^n\binom{n}{k}[-i^k+i^k] & = \binom{n}{2d+1}[-i^{2d+1}+i^{2d+1}]  \\
 & = \binom{n}{2d+1}[(-i^2)^d(-i) + (i^{2})^d(i)]  \\
 & = \binom{n}{2d+1}[(-1)^d(-i) + (-1)^d(i)]  \\
 & = \binom{n}{2d+1}[-(-1)^d(i) + (-1)^d(i)]  \\
 & = \binom{n}{2d+1}[0]  \\
 & = 0
\end{align*}
This means that for each odd k, the value of the summation will be unchanged. \\\\\\\\\\\\\\\\\\\\

In the second case we will let k be even, it can thus be expressed as  $k$ as $2d$ (where $d \in \mathbb N$), so our equation becomes:
\begin{align*}
\sum\limits_{k = 0}^n\binom{n}{k}[-i^k+i^k] & = \binom{n}{2d}[-i^{2d}+i^{2d}]  \\
 & = \binom{n}{2d}[(-i^2)^d + (i^{2})^d]  \\
 & = \binom{n}{2d}[(-1)^d + (-1)^d]  \\
 & = \binom{n}{2d}[2(-1)^d] 
\end{align*}
We thus know that only even values will impact the summation, the equation for all the even values is:
\[ \sum\limits_{k = 0}^{\lfloor n/2\rfloor} \binom{n}{2k}[2(-1)^k]  \text{ or }  2\sum\limits_{k = 0}^{\lfloor n/2\rfloor} \binom{n}{2k}[(-1)^k] \]
And because odd values are redundant we find that:
\[ (1 + i)^n + (1 - i)^n = 2\sum\limits_{k = 0}^{\lfloor n/2\rfloor} \binom{n}{2k}(-1)^k. \]

\textbf{Q04B} Let $n$ be a non negative integer, to start we can split the following into cases:
\[ (1 + i)^n + (1 - i)^n = 0 \iff n \equiv 2 \pmod 4 \]
The first case we will consider is:
\[ n \equiv 2 \pmod 4 \implies  n, (1 + i)^n + (1 - i)^n = 0\]
If we assume $n \equiv 2 \pmod 4$, this would means for some integer $j$ we could also express $n$ as:
\[ n = 4j + 2 \]
We can then plug this into our original equation:
\begin{align*}
  (1 + i)^n + (1 - i)^n & = (1 + i)^{4j+2} + (1 - i)^{4j+2}\\
 & =  (1 + i)^{4j}\cdot(1+i)^2 + (1 - i)^{4j}\cdot(1-i)^2\\
 & = (-4)^{j}\cdot(1+i)^2 + (-4)^{j}\cdot(1-i)^2\\
 & = (-4)^{j}[1+2i+i^2 + 1 - 2i +i^2]
 & = (-4)^{j}[0]
 & = 0
\end{align*}
This thus proves the first case.\\

To find the second case we will consider:
\[ (1 + i)^n + (1 - i)^n = 0 \implies n \equiv 2 \pmod 4  \]
We know from the hint that $(1 + i)^4 = (1 - i)^4 = -4$, this means that we can consider $n$ have 4 possibilities (corresponding with mod 4), let $k \in \mathbb R$:
\[ n = 4k +0, 4k + 1, 4k + 2, 4k + 3 \]
If we plug it in we get the following table:
\begin{center}
 \begin{tabular}{ c  | c } 
  4k &  4k + 1 \\ 
\hline\\
  $(1 + i)^{4j} + (1 - i)^{4j} $ & $ (1 + i)^{4j}\cdot(1+i)^1 + (1 - i)^{4j}\cdot(1-i)^1$ \\
  $(-4)^{j} + (-4)^{j} $& $ (-4)^{j}\cdot(1+i)^1 + (-4)^{j}\cdot(1-i)^1$ \\
 Not Zero &  $ (-4)^{j}[(1+i) + (1-i)]$   \\
  & $ (-4)^{j}[2]^1$  \\
  & Not Zero \\
\end{tabular}
\end{center}

\begin{center}
 \begin{tabular}{ c  | c } 
  4k+2 &  4k + 3 \\ 
\hline\\
   $ (1 + i)^{4j}\cdot(1+i)^2 + (1 - i)^{4j}\cdot(1-i)^2$ &  $ (1 + i)^{4j}\cdot(1+i)^3 + (1 - i)^{4j}\cdot(1-i)^3$ \\
   $ (-4)^{j}\cdot(1+i)^2 + (-4)^{j}\cdot(1-i)^2$& $ (-4)^{j}\cdot(1+i)^3 + (-4)^{j}\cdot(1-i)^3$ \\
  $(-4)^{j}[1+2i+i^2 + 1 - 2i +i^2] $&   $ (-4)^{j}[(1+i)^3 + (1-i)^3]$   \\
  $(-4)^{j}[0]$ & $ (-4)^{j}[-4]$\\
  Zero & Not Zero \\
\end{tabular}
\end{center}
Thus showing that the only possible solution is when$ n \equiv 2 \pmod 4$. Since both implications are proved the iff statement is also proved.

\end{document}