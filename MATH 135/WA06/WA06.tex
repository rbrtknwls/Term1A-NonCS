\documentclass[11pt]{article}
\textwidth 15cm 
\textheight 21.3cm
\evensidemargin 6mm
\oddsidemargin 6mm
\topmargin -1.1cm
\setlength{\parskip}{1.5ex}


\usepackage{amsfonts,amsmath,amssymb,enumerate}

\begin{document}
\parindent=0pt

\textbf{Robert (Robbie) Knowles MATH 135 Fall 2020: WA06}

\textbf{Q01.} To start we will use EEA to find $x$,$ y$ and $d = gcd(2996, 1520)$

\begin{center}
 \begin{tabular}{||c c c c||} 
 \hline
 x & y & r & q \\ [0.5ex] 
 \hline\hline
 1 & 0 & 2996 & 0 \\ 
 \hline
 0 & 1 & 1520 & 0 \\
 \hline
 1 & -1 & 1476 & 1 \\
 \hline
 -1 & 2 & 44 & 1 \\
 \hline
 34 &  -67 & -24 & 33 \\ 
 \hline
 -35 & 69  & 20 & 1 \\
 \hline
 69 &  -136 & -4 & 1 \\ 
 \hline
 -380 &  749 & 0 & 5\\ 
 \hline
\end{tabular}
\end{center}

According to EEA, the last second row will provide the $x$,$y$ and $d$. in other words this means that $x = 69$, $y = -136$ and $d= gcd(2996, 1520)=4$. Plugging this the original equation we find that:
\begin{align*}
2996x + 1520y &= \gcd(2996, 1520)\\
2996x + 1520y &= d \\
2996 * (69)+ 1520 * (-136) &= 4\\
4  &= 4
\end{align*}
Therefore  $x = 69$, $y = -136$ and $d= gcd(2996, 1520)=4$.\\

\textbf{Q02A} We will disprove by counter example, consider when  $a = -2$ and $b= 4$, Note that:
\begin{enumerate}
\item $\min(a, b) = \min(-2,4) = -2$.

\item$\gcd(a, b) = \gcd(-2,4) = 2$.
\end{enumerate}
However since we know that $-2 < 2$, this would imply that for  $a = -2$ and $b= 4$, $\min(a,b) < \gcd(a,b)$. Therefore since this violates the universality of the original statement, the origonal statement is disproved.

\textbf{Q02B} We will disprove by counter example, consider when  $a = 0$ and $b= 2$, Note that:
\begin{enumerate}
\item $\min(|a|, |b|) = \min(|0|,|2|) = \min(0,2) = 0$.

\item$\gcd(a, b) = \gcd(0,2) = 2$.
\end{enumerate}
However since we know that $0 < 2$, this would imply that for  $a = 0$ and $b= 2$, $\min(|a|,|b|) < \gcd(a,b)$. Therefore since this violates the universality of the original statement, the origonal statement is disproved.

\textbf{Q02C} We will prove by using case analysis, note for all cases $a,b \neq 0$ and a,b are integers.
 \textbf{The first case will be when a  = b:}
\begin{enumerate}
\item $\min(|a|, |b|) = a$

\item$\gcd(a, b) = a$
\end{enumerate}
We thus have that $\min(|a|,|b|) = \gcd(a,b)$ when  $a=b$, proving the statement for the first case.
\textbf{The second case will be when $\pmb{a > b}$ :}

\begin{enumerate}
\item $\min(|a|, |b|) = b$

\item Let d be an integer such that $\gcd(a, b) = d$
\end{enumerate}

Since d must be a common divisor of $a$ and $ b$, $d$ is bounded by:
\begin{align*}
d \leq b < a
\end{align*}
As $d$ is the $\gcd(a,b)$ and $b$ is the $\min(|a|,|b|)$ we have that:
\begin{align*}
d &\leq b < a \\
gcd(a,b) &\leq \min(|a|,|b|) < a
\end{align*}
Therefore when $ a > b$, we will have $gcd(a,b) \leq \min(|a|,|b|)$ proving the statement for the second case.
\textbf{The third case will be when $\pmb{a < b}$ :}

\begin{enumerate}
\item $\min(|a|, |b|) = a$

\item Let d be an integer such that $\gcd(a, b) = d$
\end{enumerate}

Since d must be a common divisor of $a$ and $ b$, $d$ is bounded by:
\begin{align*}
d \leq a < b
\end{align*}
As $d$ is the $\gcd(a,b)$ and $a$ is the $\min(|a|,|b|)$ we have that:
\begin{align*}
d &\leq a < b \\
gcd(a,b) &\leq \min(|a|,|b|) < b
\end{align*}
Therefore when $ a < b$, we will have $gcd(a,b) \leq \min(|a|,|b|)$ proving the statement for the third case.

Since the three cases cover all the possible values of $a$,$b$ and they are are true for each case, it must be that for all non-zero integers $a$ and $b$, $\gcd(a, b) \leq \min(|a|, |b|)$.\\


\textbf{Q03.} The implication becomes:
\begin{align*}
a \iff b
\end{align*}
\textbf{Q04.} We will prove the the binomial equation is divisible by $n+1$, by showing that for any $n\geq1$ that $n+1$ is a factor of the binomial equation
\begin{align*}
\binom{2n}{n} & = \frac{(2n)!}{n!(2n-n)!}\\
&= \frac{2n!}{n!*n!}
\end{align*}

We will divide the numerator by $n!$, notice that this will produce a series of multiplies from $(n+1)$ to $(2n)$ on the numerator and we are left with a series of multiples from $1$ to $n$ on the denominator:
\begin{align*}
\frac{2n!}{n!*n!} &= \frac{(2n)*2n-1*(2n-2)*...*(n+2)*(n+1)}{n!}\\
&= \prod_{k=1}^{n}(\frac{n+k}{k})\\
&= \prod_{k=1}^{n}(\frac{n}{k}+\frac{k}{k})\\
&= \prod_{k=1}^{n}(\frac{n}{k}+1)
\end{align*}

We will now factor out the first term (when $k = 1$) to show that if $n\geq 2$, the binomial expression will equal some integer times $(n+1)$  (we know this because all binomial coefficients are integers):
\begin{align*}
\prod_{k=1}^{n}(\frac{n}{k}+1) & = (\frac{n}{1} + 1) \prod_{k=2}^{n}(\frac{n}{k}+1)\\
&=  (n + 1) \prod_{k=2}^{n}(\frac{n}{k}+1)
\end{align*}

However if we have $n = 1$ , we will get $(n+1)$ multiplied by 1
\begin{align*}
(n + 1) \prod_{k=2}^{1}(\frac{n}{k}+1) & = (n+1)* 1
\end{align*}

Therefore as long as $ n \geq 1 $ , the binomial expression will be some integer times $n+1$. Since $n+1$ is thus a factor of the binomial equation, it will always be divisible by $n+1$.\\

\textbf{Q05.} (5 marks) Prove that, for all positive integers $d$, $m$ and $n$, if $d = \gcd(m, n)$, then $\gcd(m, nk) = \gcd(m, dk)$ for any positive integer $k$.



d = ma + nb 
m = nb + r

if m >=  n:
d = gcd(m,n)
m = qn + r
(gcd n, r)



m = nk + 
ma+nkb = ma + dkb
\end{document}