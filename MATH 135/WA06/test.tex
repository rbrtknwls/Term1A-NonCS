\documentclass[12pt]{article}
\textwidth 15cm 
\textheight 21.3cm
\evensidemargin 6mm
\oddsidemargin 6mm
\topmargin -1.1cm
\setlength{\parskip}{1.5ex}


\usepackage{amsfonts,amsmath,amssymb,enumerate}

\begin{document}



\begin{align*}
\binom{2n}{n} & = \frac{2n!}{n!(2n-n)!}\\\\
&= \frac{2n!}{n!*n!}\\\\
&= \frac{(2n)*2n-1*(2n-2)*...*(n+2)*(n+1)}{n!}\\\\
&= \prod_{k=1}^{n}\frac{n+k}{k}\\\\
&= \prod_{k=1}^{n}(\frac{n}{k}+\frac{k}{k})\\\\
&= \prod_{k=1}^{n}(\frac{n}{k}+1)\\\\
\end{align*}

We will now factor out the first term (when $k = 1$)
\begin{align*}
\prod_{k=1}^{n}(\frac{n}{k}+1) & = (\frac{n}{1} + 1) \prod_{k=2}^{n}(\frac{n}{k}+1)\\\\
&=  (n + 1) \prod_{k=2}^{n}(\frac{n}{k}+1)
\end{align*}

However if we have $n = 1$ , we will get:
\begin{align*}
(n + 1) \prod_{k=2}^{1}(\frac{n}{k}+1) & = (n+1)* 1\\\\
\end{align*}

Therefore as long as $ n > 1 $ , the binomial expression will have a factor of $n+1$. Since it has this factor, the binomial will always be divisible by $n+1$



\end{document}