\documentclass[11pt]{article}
\textwidth 15cm 
\textheight 21.3cm
\evensidemargin 6mm
\oddsidemargin 6mm
\topmargin -1.1cm
\setlength{\parskip}{1.5ex}


\usepackage{amsfonts,amsmath,amssymb,enumerate}

\begin{document}
\parindent=0pt

\textbf{Robert (Robbie) Knowles MATH 135 Fall 2020: WA06}

\textbf{Q01.} To start we will use EEA to find $x$,$ y$ and $d = gcd(2996, 1520)$

\begin{center}
 \begin{tabular}{||c c c c||} 
 \hline
 x & y & r & q \\ [0.5ex] 
 \hline\hline
 1 & 0 & 2996 & 0 \\ 
 \hline
 0 & 1 & 1520 & 0 \\
 \hline
 1 & -1 & 1476 & 1 \\
 \hline
 -1 & 2 & 44 & 1 \\
 \hline
 34 &  -67 & -24 & 33 \\ 
 \hline
 -35 & 69  & 20 & 1 \\
 \hline
 69 &  -136 & -4 & 1 \\ 
 \hline
 -380 &  749 & 0 & 5\\ 
 \hline
\end{tabular}
\end{center}

According to EEA, the last second row will provide the $x$,$y$ and $d$. in other words this means that $x = 69$, $y = -136$ and $d= gcd(2996, 1520)=4$. Plugging this the original equation we find that:
\begin{align*}
2996x + 1520y &= \gcd(2996, 1520)\\
2996x + 1520y &= d \\
2996 * (69)+ 1520 * (-136) &= 4\\
4  &= 4
\end{align*}
Therefore  $x = 69$, $y = -136$ and $d= gcd(2996, 1520)=4$.\\

\end{document}