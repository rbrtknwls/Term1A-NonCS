\documentclass[11pt]{article}
\textwidth 15cm 
\textheight 21.3cm
\evensidemargin 6mm
\oddsidemargin 6mm
\topmargin -1.1cm
\setlength{\parskip}{1.5ex}


\usepackage{amsfonts,amsmath,amssymb,enumerate}

\begin{document}


\textbf{Q03.} Given that $d=gcd(m,n)$, we can also find that $d = \gcd(m,d)$ as $m|d$. Therefore we can say that:



To start we will prove $(\{px + qy \colon x, y \in \mathbb Z\} = \mathbb Z)  \Longrightarrow ( p \neq q)$, we will use proof by contrapositive, so the function becomes:
\begin{align*}
(p = q) \Longrightarrow (\{px + qy \colon x, y \in \mathbb Z\} \neq \mathbb Z)
\end{align*}
If we assume the hypthoesis, the series:
\begin{align*}
 (\{px + qy \colon x, y \in \mathbb Z\} \neq \mathbb Z)
\end{align*}
becomes:
\begin{align*}
 (\{qx + qy \colon x, y \in \mathbb Z\} \neq \mathbb Z)\\
 (\{q(x + y) \colon x, y \in \mathbb Z\} \neq \mathbb Z)
\end{align*}
Notice that:
\begin{enumerate}
\item for every value of $q, 2 \leq q$ (as 1, 0 are not prime)

\item for every value of $x, y, (x+y)$ will be every integer
\end{enumerate}
This means that $q(x+y)$ or $qx + qy$ will be every integer expect for 1 and -1. As such this means that it is a subset of $\mathbb Z$ but not equal to $\mathbb Z$, or that:
\begin{align*}
 (\{qx + py \colon x, y \in \mathbb Z\} \subsetneq \mathbb Z)\\
\end{align*}
And so we can conclude that:
\begin{align*}
 (\{qx + py \colon x, y \in \mathbb Z\} )\neq \mathbb Z)
\end{align*}
We have thus proved the first implication of the if and only if statement.\\\\
The second impication states that: $( p \neq q) \Longrightarrow (\{px + qy \colon x, y \in \mathbb Z\} = \mathbb Z) $. Since $p$ and $q$ are unique primes such that $d = \gcd(p,q) = 1$, we can find that:
\begin{align*}
 \exists  d, a, b  \in  \mathbb Z\,, pa + bq = d = 1\\
\end{align*}
We know since $x = ad$  and $y = bd$,  that $px + qy$ becomes:
\begin{align*}
px + qy & = p(ad) + q(bd)\\
& = pad + qbd\\
& = d(ap + qb)
\end{align*}
As $d = 1$ this will become
\begin{align*}
 \exists  a, b  \in  \mathbb Z\,,(ap+qb)
\end{align*}
Since a and b can be any integer $(ap+qb) $ will equal every integer, this means that $(ap+qb) = \mathbb Z$. Note that $(ap+qb)$ will have the same domain as $ px + qy$ which means  that $ \{px + qy \colon x, y \in \mathbb Z\} = \mathbb Z$, thus proving the second implication.

Therefore since the first and second implication is true, we have proved the if and only if statement.



\end{document}