\documentclass[11pt]{article}
\textwidth 15cm 
\textheight 21.3cm
\evensidemargin 6mm
\oddsidemargin 6mm
\topmargin -1.1cm
\setlength{\parskip}{1.5ex}


\usepackage{amsfonts,amsmath,amssymb,enumerate}

\begin{document}
\textbf{Q04.} We will prove the the binomial equation is divisible by $n+1$, by showing that for any $n\geq1$ that $n+1$ is a factor of the binomial equation
\begin{align*}
\binom{2n}{n} & = \frac{(2n)!}{n!(2n-n)!}\\
&= \frac{2n!}{n!*n!}
\end{align*}

We will divide the numerator by $n!$, notice that this will produce a series of multiplies from $(n+1)$ to $(2n)$ on the numerator and we are left with a series of multiples from $1$ to $n$ on the denominator:
\begin{align*}
\frac{2n!}{n!*n!} &= \frac{(2n)*2n-1*(2n-2)*...*(n+2)*(n+1)}{n!}\\
&= \prod_{k=1}^{n}(\frac{n+k}{k})\\
&= \prod_{k=1}^{n}(\frac{n}{k}+\frac{k}{k})\\
&= \prod_{k=1}^{n}(\frac{n}{k}+1)
\end{align*}

We will now factor out the first term (when $k = 1$) to show that if $n\geq 2$, the binomial expression will equal some integer times $(n+1)$  (we know this because all binomial coefficients are integers):
\begin{align*}
\prod_{k=1}^{n}(\frac{n}{k}+1) & = (\frac{n}{1} + 1) \prod_{k=2}^{n}(\frac{n}{k}+1)\\
&=  (n + 1) \prod_{k=2}^{n}(\frac{n}{k}+1)
\end{align*}

However if we have $n = 1$ , we will get $(n+1)$ multiplied by 1
\begin{align*}
(n + 1) \prod_{k=2}^{1}(\frac{n}{k}+1) & = (n+1)* 1
\end{align*}

Therefore as long as $ n \geq 1 $ , the binomial expression will be some integer times $n+1$. Since $n+1$ is thus a factor of the binomial equation, it will always be divisible by $n+1$.\\
\end{document}