\documentclass[11pt]{article}
\textwidth 15cm 
\textheight 21.3cm
\evensidemargin 6mm
\oddsidemargin 6mm
\topmargin -1.1cm
\setlength{\parskip}{1.5ex}


\usepackage{amsfonts,amsmath,amssymb,enumerate}



\begin{document}
\parindent=0pt

\textbf{Q17.} We know the equation we need to prove is of the form:
\[ x \not\equiv [50] \text{ (mod 79)} \implies x^{11} \not\equiv [2] \text{ (mod 79)} \]
Proof by Contrapositive tells us that proving the equation's contrapositive will also prove the original equation. Thus we need to prove:
\[ x^{11} \equiv [2] \text{ (mod 79)}  \implies x \equiv [50] \text{ (mod 79)} \]
Lets start by assuming the hypothesis:
\begin{align*}
 x^{11} \equiv [2] \text{ (mod 79)} 
\end{align*}
We know from (CP), that the equation to the seventh power will equal:
\begin{align*}
 x^{77} &\equiv [2]^7 \text{ (mod 79)} \\
 x^{77} &\equiv [128] \text{ (mod 79)} \\
 x^{77} &\equiv [49] \text{ (mod 79)} 
\end{align*}
We know that x is not divisible by 79, thus Fermat's Little theorem (FLT) states: 
\[ x^{78} \equiv [1] \text{ (mod 79)} \] 
This can thus be rewritten:
\[ x^{77}\cdot x^1 = [1] \text{ (mod 79)} \]
Subbing in our value for $x^{77}$ we get:
\[ [49]x = [1] \text{ (mod 79)} \]
We know that $gcd(49,79) = 1$ and that $1|1$, therefore by LCT we know there will be solution $x$. We also know that the set of solutions will be of the form (where $x_0$ is a specific solution):
\[ x \equiv [x_0] \text{ (mod 79) } \]
Therefore to find this specific solution ($x_0$), we will rewrite our equation as: 
\begin{align*}
 49x & = 1 + 79z \text{ (For some $z \in \mathbb{Z}$)}  \\
 49x - 79z  &= 1 
\end{align*}
If we let $b = -z$, we will thus get:
\[  49x + 79b  = 1 \]
Applying EEA to this equation we get the following table:
\begin{center}
 \begin{tabular}{||c c c c||} 
 \hline
 x & b & r & q \\ [0.5ex] 
 \hline\hline
 0 & 1 & 79 & 0 \\ 
 \hline
 1 & 0 & 49 & 0 \\
 \hline
 -1 & 1 & 30 & 1 \\
 \hline
 2 & -1 & 19 & 1\\
 \hline
 -3 & 2 & 11 & 1\\ 
 \hline
 5 & -3  & 8 & 1 \\
 \hline
 -8 & 5 & 3 & 2 \\ 
 \hline
21 & -13 & 2 & 1\\ 
 \hline
-29 & 18 & 1 & 1\\ 
 \hline
\end{tabular}
\end{center}
Therefore our bottom equation tells us that our equation has a solution ($x_0$) when:
\[  x_0 = -29 \]
Thus from LCT we know our solution for x will be for the form:
\[ x \equiv [-29] \text{ (mod 79)} \]
This is equivalent to:
\begin{align*}
 x & \equiv [-29] + [79] \text{ (mod 79)} \\
 x & \equiv [50]  \text{ (mod 79)} 
\end{align*}
This thus proves the contrapositive, and as a result we have proved the original statement.
\end{document}