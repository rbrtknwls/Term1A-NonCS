\documentclass[11pt]{article}
\textwidth 15cm 
\textheight 21.3cm
\evensidemargin 6mm
\oddsidemargin 6mm
\topmargin -1.1cm
\setlength{\parskip}{1.5ex}


\usepackage{amsfonts,amsmath,amssymb,enumerate}



\begin{document}
\parindent=0pt

\textbf{Q1A.} For S to be a subset of T it would imply that every element of S belongs to T. To show this is the case, I will demonstrate that every solution to S is also solution to T.\\\\
To start we know that the solution set of S is given by (where for some $s \in \mathbb{Z}$):
\[ c = 8s \text{ (mod 12)} \]
Congruent To Remainder (CTR), tells us that:
\[ 12|(c - 8s) \]
From definition of divisibility we can rewrite this as (for some $t \in \mathbb{Z}$):
\begin{align*}
12t &= c - 8s \\
 12t - 8s &= c \\
 2(6t - 4s) &= c 
\end{align*}
Therefore we will let some integer $n = 6t - 4s$, we thus find that:
\[ 2n = c \]
By definition we find that:
\[ 2|c \]
Which shows that every solution to S is a solution to T, and thus S is a subset of T. \\\\\\
\textbf{Q1B.} Lets look at the element $e =  2$ in set T, we know that:
\[ 2|2 \]
Thus we know that $e = 2$ is a element of T. We know that the solutions (c) to S are of the form (for some integer s):
\begin{align*}
c = [8][s] \text{ (mod 12)} 
\end{align*}
Notice that $0 \leq [s] \leq 11$, thus we can construct a table that gives all possible values of c:
\begin{center}
 \begin{tabular}{||c | c c c c c c c c c c c c||} 
 \hline
 [s] & 0 & 1 & 2 & 3 & 4 & 5 & 6 & 7 & 8 & 9 & 10 & 11\\ [0.5ex] 
 \hline
 [s][8] (mod 12) & 0 & 8 & 4 & 0 & 8 &  4 & 0 & 8 & 4 & 0 & 8 & 4\\ 
 \hline
\end{tabular}
\end{center}
Therefore we notice that $e= 2$ never coursers as a solution for any value of s. Therefore we know that the set S does not contain $e=2$.\\ Thus we have shown that the element $e=2$ exists within T but not within S.
\end{document}