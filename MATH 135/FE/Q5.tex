\documentclass[11pt]{article}
\textwidth 15cm 
\textheight 21.3cm
\evensidemargin 6mm
\oddsidemargin 6mm
\topmargin -1.1cm
\setlength{\parskip}{1.5ex}


\usepackage{amsfonts,amsmath,amssymb,enumerate}



\begin{document}
\parindent=0pt

\textbf{Q5.} Let $m$ be an integer such that $m > 1$. Since $m >1$, Unique Factorization Theorem (UFT) tells us that we can express m as a product of prime factors uniquely such that (for some $z \in \mathbb{Z}$, $z \geq 1$):
\[ m = p_1^{\alpha_1}p_2^{\alpha_2}\cdots p_z^{\alpha_z}  \]
where $p_i$ (for 1 $\leq i \leq z$) represent the prime divisors of m, and where the positive integer $\alpha_i$ could be zero. This can be re-written in product notation such that:
\[ m  = \prod_{i=1}^z(p_i^{\alpha_i}) \]
We know that $\alpha_i$ is a positive integer, thus the Division Algorithm (DA) tells us that the integers $g_i$ and $r_i$ exist such that:
\[ \alpha_i = 5g_i + r_i \text { (0 $\leq r < 5$)}\]
Thus we can rewrite $n$ as:
\begin{align*}
m  &= \prod_{i=1}^z(p_i^{5g_i + r_i}) \\
m  &= \prod_{i=1}^z(p_i^{5g_i}p^r_i)\\
m  &= \prod_{i=1}^zp_i^{5g_i} \cdot \prod_{i=1}^zp^r_i\\
m  &= (\prod_{i=1}^zp_i^{g_i})^5 \cdot \prod_{i=1}^zp^r_i
\end{align*}
Using the reverse of Unique Factorization Theorem (UFT) we know we can define a positive integer $s$ to be:
\[ s = \prod_{i=1}^zp^r_i  \text { (0 $\leq r < 5$)}\]
Therefore we know that $r$ is always less then 5, so this means that s contains no fifth powers so $s$ is five-free as it cant be divisible by any fifth power.\\\\
Plugging a positive five-free integer $s$ back into our equation we get that:
\[ m  = (\prod_{i=1}^zp_i^{g_i})^5 \cdot s\]
Using the reverse of Unique Factorization Theorem (UFT) we know we can define a positive integer $t$ to be:
\[ t = \prod_{i=1}^zp_i^{q_i}  \]
Thus plugging it back into our original equation we get:
\[ m  = (t)^5 \cdot \prod_{i=1}^zp^r_i \]
where $s$ and $t$ are positive integers and $s$ is five-free.
\end{document}