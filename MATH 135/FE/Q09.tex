\documentclass[11pt]{article}
\textwidth 15cm 
\textheight 21.3cm
\evensidemargin 6mm
\oddsidemargin 6mm
\topmargin -1.1cm
\setlength{\parskip}{1.5ex}


\usepackage{amsfonts,amsmath,amssymb,enumerate}



\begin{document}
\parindent=0pt

\textbf{Q9.} For all natural numbers let $q$ and $t$, there exists integers $r$, $s$ such that:
\[ 18qu - 35tv = 1 \]
Assuming the hypothesis, we can rearrange to get that:
\[ (9q)(2u) - (7v)(5t) = 1 \]
Therefore from this equation we can know some integers ($x_0,y_0 \in \mathbb{Z}$) exist such that:
\[ x_0 = 2u \text{ and } y_0 = -5t \]
So our hypothesis equation will become:
\[ (9q)x_0 + (7v)y_0 = 1 \]
We can thus apply CCT, which tells us that since integers $x_0,y_0$ exist:
\[ \gcd(9q,7v) = 1 \]
If we go back to our hypothesis equation, we know some integers ($x_1,y_1 \in \mathbb{Z}$) exist such that:
\[ x_1 = 9q \text{ and } y_1 = -7v \]
So our hypothesis equation will become:
\[ x_1(2u) + y_1(5t) = 1 \]
We can thus apply CCT, which tells us that since integers $x_1,y_1$ exist:
\[ \gcd(5t,2u) = 1 \]
Therefore for all natural numbers $q$ and $t$, there exists integers $r$, $s$ such that:
\[ 18qu - 35tv = 1 \]
We thus know that:
\[ \gcd(9q,7v) = 1 \text{ and } \gcd(5t,2u) = 1  \]
\[ \gcd(9q,7v) = \gcd(5t,2u) \]
\end{document}