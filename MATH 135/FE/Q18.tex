\documentclass[11pt]{article}
\textwidth 15cm 
\textheight 21.3cm
\evensidemargin 6mm
\oddsidemargin 6mm
\topmargin -1.1cm
\setlength{\parskip}{1.5ex}


\usepackage{amsfonts,amsmath,amssymb,enumerate}



\begin{document}
\parindent=0pt

\textbf{18a.} To start we know that the modulus of $z_0$ is
\begin{align*}
 r_{z_0} & = \sqrt{(2)^2 + (2sqrt(3))^2} \\
& = \sqrt{4 + (4)(3)} \\
& = \sqrt{16} \\
& = 4
\end{align*}
We also know that the $\theta$ of $z_0$ is:
\begin{align*}
 \theta_{z_0} & = \tan^{-1}(\frac{2\sqrt{3}}{2})\\
 \theta_{z_0} & = \tan^{-1}(\sqrt{3})\\
 \theta_{z_0} & = \frac{\pi}{3}
\end{align*}
Thus we know $z_0$ will be:
\begin{align*}
 z_0 & = r(\cos\theta + \sin\theta i) \\
 z_0 & = 4(\cos(\frac{\pi}{3}) + \sin(\frac{\pi}{3})i) \\
\end{align*}
Moving on to $z_1$ we know its modulus is:
\begin{align*}
 r_{z_1} & = \sqrt{(-\frac{1}{\sqrt{2}})^2 + (-\frac{\sqrt{3}}{\sqrt{2}})^2 }\\
& = \sqrt{\frac{1}{2} + \frac{3}{2}} \\
& = \sqrt{\frac{4}{2}} \\
& = \sqrt{2}
\end{align*}
We also know that the $\theta$ of $z_1$ is:
\begin{align*}
 \theta_{z_1} & = \tan^{-1}(\frac{\sqrt{3}\cdot\sqrt{2}}{\sqrt{2}})\\
 \theta_{z_1} & = \tan^{-1}(\sqrt{3})\\
 \theta_{z_1} & = \frac{\pi}{3}
\end{align*}
We know from the signs that it is in third quadrant, thus it becomes:
\[ \theta_{z_1} = \frac{4\pi}{3} \]
Thus we know $z_1$ will be:
\begin{align*}
 z_1 & = r(\cos\theta + \sin\theta i) \\
 z_1 & = \sqrt{2}(\cos(\frac{4\pi}{3}) + i\sin(\frac{4\pi}{3})i) \\
\end{align*} \\\\

\textbf{18b.} To start we are given the equation:
\[ z_n = (\cos(\frac{3\pi}{4}) + \sin(\frac{3\pi}{4})i)\frac{z_{n-1}}{|z_{n-2}|} \]
We can replace $\frac{1}{|z_{n-2}|}$ with $|z_{n-2}|^{-1}$ using Properties of Modulus (PM) and our equation becomes:
\[ z_n = (\cos(\frac{3\pi}{4}) + \sin(\frac{3\pi}{4})i) \cdot z_{n-1} \cdot |z_{n-2}|^{-1} \]
However we are trying to solve for $|z_n|$ so our equation becomes:
\[ |z_n| = |(\cos(\frac{3\pi}{4}) + \sin(\frac{3\pi}{4})i) \cdot z_{n-1} \cdot |z_{n-2}|^{-1}| \]
We can use PM to distribute out the modulus so we get:
\[ |z_n| = |(\cos(\frac{3\pi}{4}) + \sin(\frac{3\pi}{4})i)| \cdot |z_{n-1}| \cdot | |z_{n-2}|^{-1}| \]
Notice that $|z|$ = $||z||$, as the imaginary component will not exist after the first modulus, so you will get (for some real number $r$:
\[ |z| = r \]
Thus:
\[ ||z|| = \sqrt{r^2 + 0^2} \]
\[ ||z|| = r \]
So our equation can be rewritten as:
\[ |z_n| = |(\cos(\frac{3\pi}{4}) + \sin(\frac{3\pi}{4})i)| \cdot |z_{n-1}| \cdot |z_{n-2}|^{-1} \]
We also know that $|(\cos(\frac{3\pi}{4}) + \sin(\frac{3\pi}{4})i)| = 1$ as its in polar form and it's $r = 1$, so:
\begin{align*}
 |z_n| &= 1 \cdot |z_{n-1}| \cdot |z_{n-2}|^{-1}  \\
|z_n| &= \frac {|z_{n-1}|}{|z_{n-2}|}  
\end{align*}
By finding the shortest non repeating sequence within z, we can find the positive integer $p$ such that for any positive integer $n$:
\[ |z_{n+p}| = |z_n| \]
To do this we will list the values of $z_n$ until we find a repeating term (we wont list $z_0$ as 0 is not a positive integer):
\begin{center}
 \begin{tabular}{||c | c||} 
\hline
 $|z_1|$ &  $\sqrt2$ (from Part A) \\ 
 \hline
 $|z_2|$ & $ \frac {|z_{n-1}|}{|z_{n-2}|} = \frac {\sqrt2}{4}$\\ 
 \hline
 $|z_3|$ & $ \frac {|z_{n-1}|}{|z_{n-2}|} = \frac {\sqrt2}{4\cdot\sqrt2} =  \frac {1}{4}$\\ 
 \hline
 $|z_4|$ & $ \frac {|z_{n-1}|}{|z_{n-2}|} = \frac {4}{4\cdot\sqrt2} =  \frac {1}{\sqrt2}$\\ 
 \hline
 $|z_5|$ & $ \frac {|z_{n-1}|}{|z_{n-2}|} = \frac {4}{\sqrt2} =  2\sqrt2$\\ 
 \hline
 $|z_6|$ & $ \frac {|z_{n-1}|}{|z_{n-2}|} = 2\sqrt2\cdot\sqrt2 =  4$\\ 
 \hline
 $|z_7|$ & $ \frac {|z_{n-1}|}{|z_{n-2}|} = \frac {4}{2\sqrt2} =  \sqrt2$\\ 
 \hline
\end{tabular}
\end{center}
Therefore we notice a pattern that for any $z_n$ where $n$ is an integer:
\[ n \equiv n_0 \text{ (mod 6)} \implies  |z_n| = |z_{n_0}| \]
Therefore be definition this tells us that for some $k$:
\[ n \equiv n + 6k \implies  |z_n| = |z_{n+6k}|\]
Let $p = 6k$, the smallest positive integer of $p$ will happen when $k = 1$. Thus:
\[ |z_n| = |z_{n+ p}| \text{ (where p = 6)} \]
\\\\\\\\\\\\\\\\\\\\\\\\\\\\\\\\\\\\\\\\
\textbf{18c} We need to find a positive integer $q$ such that for every positive integer $n, z_n =  z_{n+q}$. This will happen when both:
\[ |z_n| = |z_{n+q}| \text{ and } \arg(z_n) = \arg(z_{n+q}) \]
From Part B we know that q must be be a multiple of 6. In order to get when the arguments are equal we must come up with a general expression for $\arg(z_n)$:
\[ z_n = (\cos(\frac{3\pi}{4}) + \sin(\frac{3\pi}{4})i) \cdot z_{n-1} \cdot |z_{n-2}|^{-1} \]
We know that real components will have no impact on our argument, so we can remove $|z_{n-2}|^{-1}$, so we get:
\[ z_n = (\cos(\frac{3\pi}{4}) + \sin(\frac{3\pi}{4})i) \cdot z_{n-1} \]
We can express $z_{n-1}$ in polar form such that for some integer $r_{n-1},\theta_{n-1}$:
\[ z_{n-1} = r_{n-1}(\cos(\theta_{n-1}) + \sin(\theta_{n-1}))i) \]
Plugging this back into our equation we get that:
\[ z_n = (\cos(\frac{3\pi}{4}) + \sin(\frac{3\pi}{4})i) (r_{n-1}(\cos(\theta_{n-1}) + \sin(\theta_{n-1}))i)) \]
Thus by Polar Multiplication in C (PMC), we know that this will simplify to:
\[ z_n = r_{n-1}(\cos(\frac{3\pi}{4} +\theta_{n-1}) + \sin(\frac{3\pi}{4} + \theta_{n-1})i)  \]
And the argument of this term will thus be:
\begin{align*}
 \arg(z_n)  &= \arg[r_{n-1}(\cos(\frac{3\pi}{4} +\theta_{n-1}) + \sin(\frac{3\pi}{4} + \theta_{n-1})i)]  \\
&=  (\cos(\frac{9\pi}{12} +\theta_{n-1}) + \sin(\frac{9\pi}{12} + \theta_{n-1})i)\\
&=  \frac{9\pi}{12} +\theta_{n-1}
\end{align*}
Thus we will now find the smallest $y$ such that:
\[ \arg(z_n) = \arg(z_{n+y}) \]
In order to do this we will find the shortest non repeating sequence within z by creating a list of all $\arg(z_n)$ up to the first repeat:
\begin{center}
 \begin{tabular}{||c | c ||} 
\hline\\[-1em]
 $\arg(z_1)$ &  $\frac{4\pi}{3} = \frac{16\pi}{12}$ (from Part A) \\ \\[-1em]
 \hline \\[-1em]
 $\arg(z_2)$ &  $\frac{9\pi}{12} +\theta_{n_1} = \frac{9\pi}{12} + \frac{16\pi}{12} = \frac{1\pi}{12}$\\ \\[-1em]
 \hline \\[-1em]
 $\arg(z_3)$ & $\frac{9\pi}{12} +\theta_{n_2} = \frac{9\pi}{12} + \frac{1\pi}{12} = \frac{10\pi}{12}$\\ \\[-1em]
 \hline \\[-1em]
 $\arg(z_4)$ & $\frac{9\pi}{12} +\theta_{n_3} = \frac{9\pi}{12} + \frac{10\pi}{12} = \frac{19\pi}{12}$\\ \\[-1em]
 \hline \\[-1em]
 $\arg(z_5)$ & $\frac{9\pi}{12} +\theta_{n_4} = \frac{9\pi}{12} + \frac{19\pi}{12} = \frac{4\pi}{12}$\\ \\[-1em]
 \hline \\[-1em]
 $\arg(z_6)$ & $\frac{9\pi}{12} +\theta_{n_5} = \frac{9\pi}{12} + \frac{4\pi}{12} = \frac{13\pi}{12}$\\ \\[-1em]
 \hline \\[-1em]
$\arg(z_7)$ & $\frac{9\pi}{12} +\theta_{n_6} = \frac{9\pi}{12} + \frac{13\pi}{12} = \frac{22\pi}{12}$\\  \\[-1em]
 \hline  \\[-1em]
$\arg(z_8)$ & $\frac{9\pi}{12} +\theta_{n_7} = \frac{9\pi}{12} + \frac{22\pi}{12} = \frac{7\pi}{12}$\\  \\[-1em]
 \hline  \\[-1em]
$\arg(z_9)$ & $\frac{9\pi}{12} +\theta_{n_8} = \frac{9\pi}{12} + \frac{7\pi}{12} = \frac{16\pi}{12}$\\  
 \hline
\end{tabular}
\end{center}
Therefore we notice a pattern that for any $z_n$ where $n$ is an integer:
\[ n \equiv n_0 \text{ (mod 8)} \implies  \arg(z_n) = \arg(z_{n_0}) \]
Therefore be definition this tells us that for some $k$:
\[ n \equiv n + 8k \implies  \arg(z_n) = \arg(z_{n+9k})\]\\
We thus know that for some positive integer $q$:
\begin{align*}
 q = 0 \text{ (mod6)} & \implies |z_n| = |z_{n+q}|  \\
q = 0 \text{ (mod 8)} & \implies \arg(z_n) = \arg(z_{n+q})
\end{align*}
Thus by CTR, both of these conditions will be met (both conditions being met implies $z_n = z_{n+q}$) when:
\[ q = 0 \text{ (mod 24)} \implies z_n = z_{n+q} \]
Q is thus equal to (for some integer $t$):
\[ q = 24t \]
Thus the smallest positive integer of $q$ happens when $t = 1$ and so the smallest positive $q$ is 24. So we get:
\[ z_n = z_{n+q} \text{ where q = 24} \]
\end{document}
