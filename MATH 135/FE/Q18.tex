\documentclass[11pt]{article}
\textwidth 15cm 
\textheight 21.3cm
\evensidemargin 6mm
\oddsidemargin 6mm
\topmargin -1.1cm
\setlength{\parskip}{1.5ex}


\usepackage{amsfonts,amsmath,amssymb,enumerate}



\begin{document}
\parindent=0pt

\textbf{18a.} To start we know that the modulus of $z_0$ is
\begin{align*}
 r_{z_0} & = \sqrt{(2)^2 + (2sqrt(3))^2} \\
& = \sqrt{4 + (4)(3)} \\
& = \sqrt{16} \\
& = 4
\end{align*}
We also know that the $\theta$ of $z_0$ is:
\begin{align*}
 \theta_{z_0} & = \tan^{-1}(\frac{2\sqrt{3}}{2})\\
 \theta_{z_0} & = \tan^{-1}(\sqrt{3})\\
 \theta_{z_0} & = \frac{\pi}{3}
\end{align*}
Thus we know $z_0$ will be:
\begin{align*}
 z_0 & = r(\cos\theta + \sin\theta i) \\
 z_0 & = 4(\cos(\frac{\pi}{3}) + \sin(\frac{\pi}{3})i) \\
\end{align*}
Moving on to $z_1$ we know its modulus is:
\begin{align*}
 r_{z_1} & = \sqrt{(-\frac{1}{\sqrt{2}})^2 + (-\frac{\sqrt{3}}{\sqrt{2}})^2 }\\
& = \sqrt{\frac{1}{2} + \frac{3}{2}} \\
& = \sqrt{\frac{4}{2}} \\
& = \sqrt{2}
\end{align*}
We also know that the $\theta$ of $z_1$ is:
\begin{align*}
 \theta_{z_1} & = \tan^{-1}(\frac{\sqrt{3}\cdot\sqrt{2}}{\sqrt{2}})\\
 \theta_{z_1} & = \tan^{-1}(\sqrt{3})\\
 \theta_{z_1} & = \frac{\pi}{3}
\end{align*}
We know from the signs that it is in third quadrant, thus it becomes:
\[ \theta_{z_1} = \frac{4\pi}{3} \]
Thus we know $z_1$ will be:
\begin{align*}
 z_1 & = r(\cos\theta + \sin\theta i) \\
 z_1 & = \sqrt{2}(\cos(\frac{4\pi}{3}) + i\sin(\frac{4\pi}{3})i) \\
\end{align*} \\\\

\textbf{18b.} To start we are given the equation:
\[ z_n = (\cos(\frac{3\pi}{4}) + \sin(\frac{3\pi}{4})i)\frac{z_{n-1}}{|z_{n-2}|} \]
We can replace $\frac{1}{|z_{n-2}|}$ with $|z_{n-2}|^{-1}$ using Properties of Modulus (PM) and our equation becomes:
\[ z_n = (\cos(\frac{3\pi}{4}) + \sin(\frac{3\pi}{4})i) \cdot z_{n-1} \cdot |z_{n-2}|^{-1} \]
However we are trying to solve for $|z_n|$ so our equation becomes:
\[ |z_n| = |(\cos(\frac{3\pi}{4}) + \sin(\frac{3\pi}{4})i) \cdot z_{n-1} \cdot |z_{n-2}|^{-1}| \]
We can use PM to distribute out the modulus so we get:
\[ |z_n| = |(\cos(\frac{3\pi}{4}) + \sin(\frac{3\pi}{4})i)| \cdot |z_{n-1}| \cdot | |z_{n-2}|^{-1}| \]
Notice that $|z|$ = $||z||$, as the imaginary component will not exist after the first modulus, so you will get (for some real number $r$:
\[ |z| = r \]
Thus:
\[ ||z|| = \sqrt{r^2 + 0^2} \]
\[ ||z|| = r \]
So our equation can be rewritten as:
\[ |z_n| = |(\cos(\frac{3\pi}{4}) + \sin(\frac{3\pi}{4})i)| \cdot |z_{n-1}| \cdot |z_{n-2}|^{-1} \]
We also know that $|(\cos(\frac{3\pi}{4}) + \sin(\frac{3\pi}{4})i)| = 1$ as its in polar form and it's $r = 1$, so:
\begin{align*}
 |z_n| &= 1 \cdot |z_{n-1}| \cdot |z_{n-2}|^{-1}  \\
|z_n| &= \frac {|z_{n-1}|}{|z_{n-2}|}  
\end{align*}
By finding the shortest non repeating sequence within z, we can find the positive integer $p$ such that for any positive integer $n$:
\[ |z_{n+p}| = |z_n| \]


\end{document}
