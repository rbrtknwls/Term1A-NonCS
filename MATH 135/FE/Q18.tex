\documentclass[11pt]{article}
\textwidth 15cm 
\textheight 21.3cm
\evensidemargin 6mm
\oddsidemargin 6mm
\topmargin -1.1cm
\setlength{\parskip}{1.5ex}


\usepackage{amsfonts,amsmath,amssymb,enumerate}



\begin{document}
\parindent=0pt

\textbf{18a.} To start we know that the modulus of $z_0$ is
\begin{align*}
 r_{z_0} & = \sqrt{(2)^2 + (2sqrt(3))^2} \\
& = \sqrt{4 + (4)(3)} \\
& = \sqrt{16} \\
& = 4
\end{align*}
We also know that the $\theta$ of $z_0$ is:
\begin{align*}
 \theta_{z_0} & = \tan^{-1}(\frac{2\sqrt{3}}{2})\\
 \theta_{z_0} & = \tan^{-1}(\sqrt{3})\\
 \theta_{z_0} & = \frac{\pi}{3}
\end{align*}
Thus we know $z_0$ will be:
\begin{align*}
 z_0 & = r(\cos\theta + i\sin\theta) \\
 z_0 & = 4(\cos(\frac{\pi}{3}) + i\sin(\frac{\pi}{3})) \\
\end{align*}
Moving on to $z_1$ we know its modulus is:
\end{document}