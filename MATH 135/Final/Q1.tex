\documentclass[11pt]{article}
\textwidth 15cm 
\textheight 21.3cm
\evensidemargin 6mm
\oddsidemargin 6mm
\topmargin -1.1cm
\setlength{\parskip}{1.5ex}


\usepackage{amsfonts,amsmath,amssymb,enumerate}



\begin{document}
\parindent=0pt

\textbf{Q05} To start we are given that:
\[ \lim_{x \to 4} f(x) = 7 \]
From our $\epsilon-\delta$ definition we can rewrite this as:
\[ \text{if } 0 < |x - 4| < \delta \text{ then } |f(x) - 7| < \epsilon \]
We can further expand out our equation for epsilon:
\begin{align*}
|f(x) - 7|  & < \epsilon\\
7 - \epsilon <  f(x)  & < 7 + \epsilon\\
15 - \epsilon <  f(x) + 8  & < 15 + \epsilon\\
|f(x) + 8|  & < \epsilon + 15
\end{align*}
So we could rewrite the given arbitrary equation as:
\[ |f(x) + 8||f(x) - 7| < (15 + \epsilon)\epsilon \]
We know that $ (15 + \epsilon)\epsilon$ will result in a value that's a little larger then the original epsilon, so we can define a new epsilon ($\epsilon'$) that's equal to it such that:
\[ \epsilon' = (15 + \epsilon)\epsilon  \]
This results in the previous equation becoming:
\begin{align*}
|f(x) + 8||f(x) - 7| & <\epsilon' \\
|(f(x))^2 + f(x) - 56| & <\epsilon'  \\
|(f(x))^2 + f(x) + 1 - 57| & <\epsilon' 
\end{align*}
This is in the form $|f(x) -L| < \epsilon' $, so by the  $\epsilon-\delta$ definition it will become: 
\begin{align*}
 \lim_{x \to 4}[(f(x))^2 + f(x) + 1] & = 57
\end{align*}
\end{document}