\documentclass[11pt]{article}
\textwidth 15cm 
\textheight 21.3cm
\evensidemargin 6mm
\oddsidemargin 6mm
\topmargin -1.1cm
\setlength{\parskip}{1.5ex}


\usepackage{amsfonts,amsmath,amssymb,enumerate}



\begin{document}
\parindent=0pt

\textbf{Q03a} We know that $f'(x)$ is defined for all $x \in \mathbb{R}$, and we also know that differentiability applies continuity for $f(x)$. This means that f(x) is both continuous and differentiable for all $x \in \mathbb{R}$.\\
Therefore we can use the bounded derivative theorem (BDT) to find the smallest interval of f(6), and since we know that:
\[ 1 \leq f'(x) \leq 5 \text{ for every } x \in (3,6) \]
Therefore by BDT this implies:
\begin{align*}
f(3) + 1(x-3)  & \leq f(x) \leq f(3) + 5(x-3) \\
1 + 1(x-3)  & \leq f(x) \leq 1+ 5(x-3) \\
1 + x-3  & \leq f(x) \leq 1+ 5x-15 \\
x - 2  & \leq f(x) \leq 5x-14 \\
\end{align*}
Therefore if we let x = 6, we find that:
\begin{align*}
6 - 2  & \leq f(6) \leq 30-14 \\
4  & \leq f(6) \leq 16 
\end{align*}
We can thus say that the smallest interval that $f(6)$ is bounded by is [4,16]\\


\textbf{Q03b} The textbook defines the second order Taylor Polynomial as (for some real x = a):
\[ T_{2,a}= \sum_{k=0}^{2}\frac{f^{(k)}(a)}{k!}(x-a)^k \]
If we expand this out we get that:
\[ T_{2,a}= \frac{f(a)}{0!}(x-a)^0 + \frac{f'(a)}{1!}(x-a)^1 + \frac{f''(a)}{2!}(x-a)^2  \]
Since we are finding this polynomial centered at x = 3, the polynomial becomes:
\[ T_{2,3}= \frac{f(3)}{0!}(x-3)^0 + \frac{f'(3)}{1!}(x-3)^1 + \frac{f''(3)}{2!}(x-3)^2  \]
From definition we know that $f(3) = 1$, $f'(3) = 5$ and $f''(3) = 7$, so we can simplify:
\begin{align*}
T_{2,3}& = \frac{1}{0!} + \frac{5}{1!}(x-3)^1 + \frac{7}{2!}(x-3)^2 \\
T_{2,3}& = 1 + 5(x-3) + \frac{7}{2}(x^2-6x+9) \\
T_{2,3}& = 1 + 5x - 15 + \frac{7x^2}{2} - \frac{42x}{2} +\frac{63}{2}  \\
T_{2,3}& = 1 + 5x - \frac{30}{2} + \frac{7x^2}{2} - 21x +\frac{63}{2}  \\
T_{2,3}& = 1 + \frac{7x^2}{2} - 16x +\frac{33}{2}  \\
\end{align*}
\end{document}