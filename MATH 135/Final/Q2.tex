\documentclass[11pt]{article}
\textwidth 15cm 
\textheight 21.3cm
\evensidemargin 6mm
\oddsidemargin 6mm
\topmargin -1.1cm
\setlength{\parskip}{1.5ex}


\usepackage{amsfonts,amsmath,amssymb,enumerate}



\begin{document}
\parindent=0pt

\textbf{Q02a} Let b be a real number. We know from the "Critical Point" Theorem (Page 202) that a critical point will happen at some c when:
\[ f'(c) = 0 \text{ or } f'(c) = DNE \]
Thus in order to solve we must first find the first derivative.
\begin{align*}
f(x) & = x^{\frac{1}{3}} + bx^{\frac{4}{3}}\\
\frac{dy}{dx}(f(x)) & = \frac{dy}{dx}(x^{\frac{1}{3}} + bx^{\frac{4}{3}})\\
f'(x) & = \frac{dy}{dx}(x^{\frac{1}{3}}) + \frac{dy}{dx}(bx^{\frac{4}{3}})\\
f'(x) & = \frac{1}{3}x^{(\frac{1}{3}-1)} + b\frac{4}{3}x^{(\frac{4}{3} -1)}\\
f'(x) & = \frac{1}{3x^{\frac{2}{3}}} + b\frac{4}{3}x^{\frac{1}{3}}
\end{align*}
Since we can see an $x$ is in the denominator for the first term, this would imply that f'(x) is undefined at 0 or:
\[ f'(0) = DNE \]
Since there is only one possible value such that$ f'(c) = DNE$, we will now try to solve $f'(c) = 0$, starting by setting f'(x) to be zero:
\begin{align*}
0 & = \frac{1}{3x^{\frac{2}{3}}} + b\frac{4}{3}x^{\frac{1}{3}} \\
 -\frac{1}{3x^{\frac{2}{3}}} & = b\frac{4}{3}x^{\frac{1}{3}} \\
 -\frac{1}{3x^{\frac{2}{3}}}x^{\frac{2}{3}} & = b\frac{4}{3}x^{\frac{1}{3}}x^{\frac{2}{3}} \\
 -\frac{1}{3} & = b\frac{4}{3}x \\
 -\frac{1}{4b} & = x
\end{align*}
Therefore our possible critical points will be when:
\[ x = 0 \text{ or }  -\frac{1}{4b} \] \\\\\\\\\\\\\\

\textbf{Q02b} We know we have two possible values to test if a local minimum can exist, point 0 and point $-\frac{1}{4b}$ (for some real number b). At the end I also prove this using an interval table. \\\\
To see if 0 in a minimum we shall use "The First Derivative Test" (Page 223)  we will use the interval $(0^-,0^+)$ to check if:
\begin{align*}
f'(0^-)<f'(x) < 0 &\text{ for all $x \in (0^-,0)$} \\
f'(0^+)>f'(x) > 0 &\text{ for all $x \in (0,0^+)$}
\end{align*}
Evaluating the first equation we find a contradiction at $f'(0^-)$ we get:
\begin{align*}
f'(0^-)< f'(x) < 0 \\
\frac{1}{3(0^-)^{\frac{2}{3}}} + b\frac{4}{3}(0^-)^{\frac{1}{3}}< f'(x) < 0 \\
\frac{1}{3(0^+)} + b\frac{4}{3}(0^-)< f'(x) < 0 \\
\frac{1}{(0^+)} + (0^-)< f'(x) < 0 \\
\infty + (0^-)< f'(x) < 0 \\
\infty < f'(x) < 0 
\end{align*}
Because this contradicts the First Derivative Test this means that 0 can not be a local minimum (as well this follows logically as the value to the will always be smaller left is smaller).
\begin{align*}
f'(x) & = \frac{1}{3x^{\frac{2}{3}}} + b\frac{4}{3}x^{\frac{1}{3}}\\
\end{align*}
Moving on we can test the critical point $-\frac{1}{4b}$ using the "Second Derivative Test", in order to solve this we will first find the second derivative.
\begin{align*}
f'(x) & = \frac{1}{3}x^{-\frac{2}{3}} + b\frac{4}{3}x^{\frac{1}{3}}\\
\frac{dy}{dx}f'(x) & = \frac{dy}{dx}(\frac{1}{3}x^{-\frac{2}{3}}) + \frac{dy}{dx}(b\frac{4}{3}x^{\frac{1}{3}})\\
f''(x) & = \frac{1}{3}\frac{-2}{3}x^{-\frac{5}{3}} + b\frac{4}{3}\frac{1}{3}x^{-\frac{2}{3}}\\
f''(x) & = \frac{-2}{9}x^{-\frac{5}{3}} + b\frac{4}{9}x^{-\frac{2}{3}}\\
\end{align*}
Now we will plug in the critical point (c) and simplify:
\begin{align*}
f''(c) & = \frac{-2}{9}(-\frac{1}{4b})^{-\frac{5}{3}} + b\frac{4}{9}(-\frac{1}{4b})^{-\frac{2}{3}}\\
f''(c) & = \frac{2}{9}(-\frac{1}{4b})^{-\frac{2}{3}}(-(-\frac{1}{4b})^{-1}+2b) \\
f''(c) & = \frac{2}{9}(-4b)^{\frac{2}{3}}(-(-4b)^{1}+2b) \\
f''(c) & = \frac{2}{9}\sqrt[\leftroot{-2}\uproot{2}3]{(16b^2)}(6b) 
\end{align*}
The "Second Derivative Test" tells us that a minimum will exist at the critical point if:
\[ f''(c) > 0 \]
As we can see from the above equation, the positivity is of the equation is solely reliant on the $6b$ term as ($b^2$ will always be positive). Therefore f'(c) > 0 if b is also positive. In other words the minimum will exist as along as:
\[ b > 0 \]
Therefore we have shown that the only possible local exist when b is positive and the minimum will exists at:
\[ -\frac{1}{4b} \]
\end{document}