\documentclass[11pt]{article}
\textwidth 15cm 
\textheight 21.3cm
\evensidemargin 6mm
\oddsidemargin 6mm
\topmargin -1.1cm
\setlength{\parskip}{1.5ex}


\usepackage{amsfonts,amsmath,amssymb,enumerate}



\begin{document}
\parindent=0pt
\textbf{Q05} Let $x$ be an integer and let $p$ be an odd prime, we can thus start with an if and only if statement which can be expressed as:
\[ x^2+1 \equiv 0 \text{ (mod $p^2$)}  \iff x^2+1 \equiv 0 \text{ (mod p)}\]
We will split this into two cases such that the first case is:
\[ x^2+1 \equiv 0 \text{ (mod $p^2$)}  \implies x^2+1 \equiv 0 \text{ (mod p)}\]
To start we will assume the hypothesis such that $x_0^2+1 \equiv 0$ (mod $p^2$) has a solution $x_0$, CTR tells us that therefore:
\[ p^2 | (x_0^2 + 1) \]
We know from TD that since  $p^2 | (x_0^2 + 1)$ and $p|p^2$ that:
\[ p | (x_0^2 + 1) \]
Thus from CTR we find that:
\[ x_0^2 + 1 = 0 \text{ (mod p)} \]
This thus proves the first iimplication as if a solution exists in the hypothesis that same solution (and others) will also exist in the conclusion.

The second case is of the form:
\[ x^2+1 \equiv 0 \text{ (mod $p$)}  \implies x^2+1 \equiv 0 \text{ (mod $p^2$)}\]
To start we will assume the hypothesis such that $x_0^2+1 \equiv 0$ (mod $p$) has a solution, we also know that x can be re-written as:
\begin{align*}
x^2+1 \equiv 0 \text{ (mod $p$)} & = (qp+r)^2 + 1  \equiv \text{ (mod $p$)} \\
\end{align*}
Expanding we find that  it equals:
\[ q^2p^2 +2qpr + r^2 + 1  \equiv 0 \text{ (mod $p$)}\]
Lets get rid of the left most term by using CAM (as we know $p^2$ (mod p) = 0):
\[ q^2(0)^2 +2qpr + r^2 + 1  \equiv 0 \text{ (mod $p$)}\]
\[ 2qpr + r^2 + 1  \equiv 0 \text{ (mod $p$)}\]
We will let a value $s$ such that s is a solution of $s^2 + 1 = 0$ (mod p), from CTR this becomes $p|(s^2+1)$ and for some integer o, $po = s^2+1$. Setting r = s we find that:
\[ 2qps + s^2 + 1  \equiv 0 \text{ (mod $p$)}\]
\[ 2qps + (s^2 + 1)  \equiv 0 \text{ (mod $p$)}\]
\[ 2qps +po \equiv 0 \text{ (mod $p$)}\]
If we apply CTR again we find that:
\[ p | 2qps +po \]
In order for there to be a solution for \[ 2qps +po \equiv 0 \text{ (mod $p$)}\], the CTR will become:
\[ p^2 | p(2qs +o) \]
\[ p |2qs +o \]
The definition of derivatives tells us that for some integer m:
\[ 2qs + o = mp \]
The solution set to $p^2 | p(2qs +o)$ , is thus equal to $2qs + o = mp$ which now has the congurence form:
\[(2s)q = -o \text{ (mod p)}\]
Notice that $2s$ is even while $p$ is false this implies that from LCT, gcd(2s,p) = 1 and $1|-o$ (as m is defined as an integer). This means a solution exists to $(2s)q = -o \text{ (mod p)}$ if we have a solution to  $x_0^2+1 \equiv 0$ (mod $p$). \\More over we know the solution set of $(2s)q = -o \text{ (mod p)}$ is equal to the solution set of \[ p^2 | p(2qs +o) \] and since by CTR thats equal to:
\[ 2pqs +po = 0 \text{ (mod $p^2$)} \]\\
We have thus proved this has a solution if  $x_0^2+1 \equiv 0$ (mod p) has a the solution $x$, thus proving the implication. Now that we have proved both implications true for any $p$ and the solution $x$ is an integer (as q,b,r are all integers) we have proved the if and only if statement.
\end{document}



