\documentclass[11pt]{article}
\textwidth 15cm 
\textheight 21.3cm
\evensidemargin 6mm
\oddsidemargin 6mm
\topmargin -1.1cm
\setlength{\parskip}{1.5ex}


\usepackage{amsfonts,amsmath,amssymb,enumerate}



\begin{document}
\parindent=0pt


\textbf{Q03a} We will prove the results by using strong induction on $n$, where $P(n)$ is the statement:
\[ a_{n+2} = 6a_{n} \]
Base Case: When n is $0$, the statement $P(0)$ is given by:
\[ x_2 \equiv 6x_0 \text{ (mod 19)} \]
This is from definition and satifies the general equation, thus proving $P(0)$.
When $n$ is 1, the statement $P(1)$ is given by:
\[ x_3 \equiv 6x_1 \text{ (mod 19)} \]
This is also from definition and satifies the general equation thus proving $P(1)$.
When $n$ is 2, the statement $P(2)$ is given by:
\begin{align*}
x_4 & = x_3 + x_2 \text{ (mod 19)} \\
& = (6x_1) + (6x_0) \text{ (mod 19)} \\
& =  6(x_1 + x_0) \text{ (mod 19)} \\
& =  6(x_2) \text{ (mod 19)} \\
\end{align*}
We have thus shown that when n is 2, it satifies the general equation thus proving $P(1)$.
Inductive Hypothesis: Let $k$ be a positive integer such that ($k \geq  2$). Assume for all integers $i$ = 1,2,3,...,$k$,  $a_{i+2} = 6 a_{i}$.
We wish to prove that when P(k+1), we will get:
\[ x_{(k+3} = 6{k+1} \]
We know that from our starting definition P(k+1):
\begin{align*}
x_{(k+1) + 2} & = x_{(k+1) + 2-1}  +x_{(k+1) + 2-2}  \text{ (mod 19)} \\
x_{k+3} & = x_{k+2}  +x_{k+1} \text{ (mod 19)} \\
\end{align*}
From our Inductive Hypthoesis we assume P(k) and P(k-1) are true, this means also due to CAM that $a_{k+2} = 6a_k$
& =  6(x_2) \text{ (mod 10)} \\
\end{align*}
\end{document}