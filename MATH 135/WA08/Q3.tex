\documentclass[11pt]{article}
\textwidth 15cm 
\textheight 21.3cm
\evensidemargin 6mm
\oddsidemargin 6mm
\topmargin -1.1cm
\setlength{\parskip}{1.5ex}


\usepackage{amsfonts,amsmath,amssymb,enumerate}



\begin{document}
\parindent=0pt


\textbf{Q03a} We will prove the results by using strong induction on $n$, where $P(n)$ is the statement:
\[ a_{n+2} = 6a_{n} \]
Base Case: When n is $0$, the statement $P(0)$ is given by:
\[ a_2 \equiv 6a_0 \text{ (mod 19)} \]
This is from definition and satifies the general equation, thus proving $P(0)$.
When $n$ is 1, the statement $P(1)$ is given by:
\[ a_3 \equiv 6a_1 \text{ (mod 19)} \]
This is also from definition and satifies the general equation thus proving $P(1)$.
When $n$ is 2, the statement $P(2)$ is given by:
\begin{align*}
a_4 & = a_3 + a_2 \text{ (mod 19)} \\
& = (6a_1) + (6a_0) \text{ (mod 19)} \\
& =  6(a_1 + a_0) \text{ (mod 19)} \\
& =  6(a_2) \text{ (mod 19)} \\
\end{align*}
We have thus shown that when n is 2, it satifies the general equation thus proving $P(1)$.
Inductive Hypothesis: Let $k$ be a positive integer such that ($k \geq  2$). Assume for all integers $i$ = 1,2,3,...,$k$,  $a_{i+2} = 6 a_{i}$.
We wish to prove that when P(k+1), we will get:
\[ a_{k+3} = a_{k+1} \text{ (mod 19)} \]
We know that from our starting definition P(k+1):
\begin{align*}
a_{(k+1) + 2} & = a_{(k+1) + 2-1}  ax_{(k+1) + 2-2}  \text{ (mod 19)} \\
a_{k+3} & = a_{k+2}  +a_{k+1} \text{ (mod 19)} \\
\end{align*}
From our Inductive Hypthoesis we assume P(k) and P(k-1) are true, this means also that $a_{k+2} = 6a_k$ (P(k)) and $a_{k+1} = 6a_k$ (P(k-1)) hold and are true. Thus our equation becomes (thanks to CAM):
\begin{align*}
a_{k+3} &= a_{k+2}  +a_{k+1} \text{ (mod 19)} \\ 
 &= 6a_{k}  +6a_{k-1} \text{ (mod 19)} \\ 
 &= 6(a_{k}+a_{k-1)} \text{ (mod 19)} \\
&= 6a_{k+1} \text{ (mod 19)} 
\end{align*}
Thus since both the base case and inductive step hold, it follows by the Principle of Strong Mathematical Induction (POSI) that the statement it true.\\\\\\\\\\\\\
\textbf{Q03b} We know from the fundimental definition given in the question that:
\begin{align*}
a_{3} &= a_{2}  +a_{1} \text{ (mod 19)} \\ 
 &= 6a_0  +a_{1} \text{ (mod 19)} 
\end{align*}
However we also know that:
\[ a_{3}= 6a_{1} \text{ (mod 19)} \]
So combining the two we find that:
\[ 6a_{1} \text{ (mod 19)}  = 6a_0 - a_1 \text{ (mod 19)} \]
\[ 0 = 6a_0 - 5a_1 \text{ (mod 19)} \]
All the possible interger combinations are given by the table:

\begin{center}
 \begin{tabular}{||c c c c||} 
 \hline
$a_0$ & $a_1$ & r & r (mod 19) \\ [0.5ex] 
 \hline\hline
 0 & 0 & 0 & 0 \\ 
 \hline
 1 & 5 & -19 & 0 \\
 \hline
 2 & 10 & -38 & 0 \\
 \hline
 3 & -15 & -57 & 0\\
 \hline
 5 &  6 & 0 & 0 \\ 
 \hline
 7 & 16  & -38 & 0 \\
 \hline
 11 & 17 & -19 & 0 \\ 
 \hline
13 & 8 & 38 & 0\\ 
 \hline
17 &  9 & 57 & 0\\ 
 \hline
\end{tabular}
\end{center}
Notice that any composite number can be created by multiplying $a_0$ and $a_1$ by some integer but the $r$ (mod 19) will still be zero. Therefore all the prime and composite numbers are in the solution set so therefore their are 19 integer pairs which act as the solution

\end{document}