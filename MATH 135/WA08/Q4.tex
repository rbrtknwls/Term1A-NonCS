\documentclass[11pt]{article}
\textwidth 15cm 
\textheight 21.3cm
\evensidemargin 6mm
\oddsidemargin 6mm
\topmargin -1.1cm
\setlength{\parskip}{1.5ex}


\usepackage{amsfonts,amsmath,amssymb,enumerate}



\begin{document}
\parindent=0pt


\textbf{Q04a} Let $a$ be an arbitrary integer, we start with an if and only if statement which can be expressed as:
\[ 37|a \iff= 37|S(a) \]
To prove this we will split it up by cases starting with:
\[ 37|a \implies 37|S(a) \]
Let assume that $37|a$, this would mean that from CTR:
\[ a \text{ (mod 37)} = 0  \text{ (mod 37)} \]
We also know that $a$ can be expressed in terms of its digits (where $k$ is an integer and $3|k$, if any digit $d_k$ is greater then what exists in $a$ then $d_k = 0$) :
\begin{align*}
0 \text{ (mod 37)} & = a \text{ (mod 37)}  \\
& = d_k(10^{k-1}) + d_{k-1}(10^{k-2}) + ... + d^2(10^2) + d^1(10) +d_0 \text{ (mod 37)}  
\end{align*}
We notice that the digit expression (mod 37) must also be equal to 0, from CAM. We also notice the following pattern in terms of the coefficents of 10 ($n$) and (mod 37):
\[ (n = 0): (10^{0}) \text{ (mod 37)} \implies 1 \text{ (mod 37)} \implies 1 \text{ (mod 37)}  \]
\[ (n = 1): (10^{1}) \text{ (mod 37)} \implies 10 \text{ (mod 37)} \implies 10 \text{ (mod 37)}  \]
\[ (n = 2): (10^{2}) \text{ (mod 37)} \implies 100 \text{ (mod 37)} \implies 26 \text{ (mod 37)} \]
\[ (n = 3): (10^{3}) \text{ (mod 37)} \implies 1000 \text{ (mod 37)} \implies 1 \text{ (mod 37)} \]
Notice that this pattern repeats. Returning back to our digit expansion, we know we can group terms into threes like:
\begin{align*}
0 \text{ (mod 37)} & = d_k(10^{k-1}) + d_{k-1}(10^{k-2})+ d_{k-2}(10^{k-3}) + ... + d_2(10^2) + d_1(10) +d_0 \text{ (mod 37)}  \\
& = (d_k(10^2) + d_{k-1}(10^1) + d_{k-2}(10^0))(10^{k-3}) +... + (d^2(10^2) + d_1(10) +d_0)(10^0) \text{ (mod 37)}  
\end{align*}
Since $3|k$ (so for some integer $a$, $3a = k$) this can be better expressed in sigma notation where:
\begin{align*}
0 \text{ (mod 37)} & =  \sum\limits_{n = 0}^a 10^{3n} (d_{2+3n}(10^2) + d_{1+3n}(10) +d_{3n})  \text{ (mod 37)}   \\
& = \sum\limits_{n = 0}^a (10^3)^n (d_{2+3n}(10^2) + d_{1+3n}(10) +d_{3n})  \text{ (mod 37)}   \\
& = \sum\limits_{n = 0}^a (1)^n (d_{2+3n}(10^2) + d_{1+3n}(10) +d_{3n})  \text{ (mod 37)}   \\
& = \sum\limits_{n = 0}^a  (d_{2+3n}(10^2) + d_{1+3n}(10) +d_{3n})  \text{ (mod 37)}   
\end{align*}
Thus we can see that our equation is equivilent to S(a), and thus:
\begin{enumerate}
\item $  S(a) = \sum\limits_{n = 0}^a  (d_{2+3n}(10^2) + d_{1+3n}(10) +d_{3n})  \text{ (mod 37)} $
\item $  \sum\limits_{n = 0}^a  (d_{2+3n}(10^2) + d_{1+3n}(10) +d_{3n})  \text{ (mod 37)} = 0 \text{ (mod 37)}  $
\end{enumerate}
This infers that S(a) = 0 (mod 37), which because of CTR proves that $37|S(a)$. We have thus proved one the first implication.\\\\
We will now try to prove the second case:
\[ 37|S(a) \implies 37|a \]
Let $37|S(a)$ this means that:
\[ S(a) \text{ (mod 37)} = 0  \text{ (mod 37)} \]
We know that we can express S(a) as:
 \[ 0 \text{ (mod 37)} = \sum\limits_{n = 0}^a  (d_{2+3n}(10^2) + d_{1+3n}(10) +d_{3n})  \text{ (mod 37)} \]\
We know from CAM that we can multiply both sides by 1 (mod 37), which could then become:
\begin{align*}
0 \text{ (mod 37)} & = \sum\limits_{n = 0}^a (1)^n (d_{2+3n}(10^2) + d_{1+3n}(10) +d_{3n})  \text{ (mod 37)}   \\
& = \sum\limits_{n = 0}^a (10^3)^n (d_{2+3n}(10^2) + d_{1+3n}(10) +d_{3n})  \text{ (mod 37)}   \\
& =  \sum\limits_{n = 0}^a 10^{3n} (d_{2+3n}(10^2) + d_{1+3n}(10) +d_{3n})  \text{ (mod 37)}   
\end{align*}
Expanding out the sigma notation we get that:
\[ 0 \text{ (mod 37)} = d_k(10^{k-1}) + d_{k-1}(10^{k-2}) + ... + d^2(10^2) + d^1(10) +d_0 \text{ (mod 37)}  \]
\[ 0 \text{ (mod 37)} = a  \]
Therefore CTR shows us that $37|a$ we have thus proved the second implications. Since we have proven both implication for all possible values of $a$, we have proved the if and only if statements.

\textbf{Q04b} Notice that in order for the previous if and only if statement to be true, you need a repeating pattern where $10^3$ (mod 27) = 1 plus $10^2, 10^1$ and $10^0$ need to have unquie endings. Thus if we plug in the numbers 0-4 we should notice a similar pattern to the (mod 37') pattern:
\[ (n = 0): (10^{0}) \text{ (mod 27)} \implies 1 \text{ (mod 27)} \implies 1 \text{ (mod 27)}  \]
\[ (n = 1): (10^{1}) \text{ (mod 27)} \implies 10 \text{ (mod 27)} \implies 10 \text{ (mod 27)}  \]
\[ (n = 2): (10^{2}) \text{ (mod 27)} \implies 100 \text{ (mod 27)} \implies 19 \text{ (mod 27)} \]
\[ (n = 3): (10^{3}) \text{ (mod 27)} \implies 1000 \text{ (mod 27)} \implies 1 \text{ (mod 27)} \]
Since $10^3$ is 1, this means that you could get rid of $10^{3n}$ term, which along with the previous conditions and CAM means that if question 1's mod was replaced with 27 the if and only if statement would be true. 
\end{document}