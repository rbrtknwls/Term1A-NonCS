\documentclass[11pt]{article}
\textwidth 15cm 
\textheight 21.3cm
\evensidemargin 6mm
\oddsidemargin 6mm
\topmargin -1.1cm
\setlength{\parskip}{1.5ex}


\usepackage{amsfonts,amsmath,amssymb,enumerate}



\begin{document}
\parindent=0pt


\textbf{Q02} Let $a$ be a positive integer, note that any $a$ can be expressed as:
\[ a = a_{rest}(10) + a_0  \] 
 where $a_0$ is the last digit and $a_{rest}$ is all the other digits. If we take the apply (mod 10) we thus get:
\begin{align*}
a \text{ (mod 10)} & \equiv  a_{rest}(10) + a_0  \text{ (mod 10)} \\
& \equiv  a_{rest}(0) + a_0  \text{ (mod 10)} \\
& \equiv  a_0  \text{ (mod 10)} \\
\end{align*}
Thus we have shown that for any real number $a$, the last digit of $a$ will be equal to $a$ (mod 10). Since it has the same functionality as D(a) (as a is a positive integer) we can write D(a) as:
\[ D(a) = a_0 \text{ (mod 10)}  \] 
From this, we can see that for an arbitrary natural m our original equation will become:
\[  D(19^m + \sum\limits_{k = 1}^m k!) = 19^m + \sum\limits_{k = 1}^m k! \text{ (mod 10)} \]
CAM tells us this equation can be split into two parts:
\begin{enumerate}
\item $19^m   \text{ (mod 10)}$
\item $\sum\limits_{k = 1}^m k! \text{ (mod 10)}$
\end{enumerate}
Starting with [1] we can see that for every natural $m$, the equation becomes:
\begin{align*}
19^m \text{ (mod 10)} & = 9^m \text{ (mod 10)} \\
 & = 9^{m-1}(9) \text{ (mod 10)}
\end{align*}
If we apply CAM, this asserts that the last digit of $9^m$ is equal to the last digit of $9^{m-1}9$, Plugging in 1 (odd) and 2 (even) for $m$ we find that:
\[ (m = 1): (9^{1-1})(9) \text{ (mod 10)} \implies (1)(9) \text{ (mod 10)} \implies 9 \]
\[ (m = 2): (9^{2-1})(9) \text{ (mod 10)} \implies (9)(9) \text{ (mod 10)} \implies 1 \]
Lets split m into two cases, if $m$ is even then by definition an integer $a$ exists such that $2a = m$. Our equation thus becomes:
\begin{align*}
 9^m \text{ (mod 10)} & = (9^2a) \text{ (mod 10)}\\
& =(9^2)^a \text{ (mod 10)}\\
& =(1)^a \text{ (mod 10)}\\
& =1 \text{ (mod 10)}
\end{align*}
if $m$ is odd then by definition an integer $a$ exists such that $2a = m - 1$. Our equation thus becomes:
\begin{align*}
9^m \text{ (mod 10)} & = 9^{m-1}(9) \text{ (mod 10)}\\
& = 9^{2a}(9) \text{ (mod 10)}\\
& = (9^2)^a(9) \text{ (mod 10)}\\
& =1^a(9) \text{ (mod 10)}\\
& = 9 \text{ (mod 10)}
\end{align*}
Thus we can see, if m is odd then [1] is 9. On the other hand if $m$ is even then [1] is 1.\\\\
Moving on to [2], we will again split the equation into cases, if $m$ is greater then 5 we see the equation will become:
\[ \sum\limits_{k = 1}^m k! \text{ (mod 10)} \equiv 1! + 2! + 3! + 4! +5!+ ...+  m! \text{ (mod 10)}\]
Notice however that for any $(k \geq 5)$, that k! (because of CAM) will be equal to:
\begin{align*}
k! & = 1 \times 2 \times 3 \times 4 \times 5 \times  ... \times  (k-1) \times (k) \ \text{ (mod 10)}\\
 & = (2 \times 5)(1 \times 3 \times 4 \times  ... \times  (k-1) \times (k)) \ \text{ (mod 10)}\\
& =(10)(1 \times 3 \times 4 \times  ... \times  (k-1) \times (k)) \ \text{ (mod 10)}\\
& =(0)(1 \times 3 \times 4 \times  ... \times  (k-1) \times (k)) \ \text{ (mod 10)}\\
& =0
\end{align*}
Thus for any $(m \geq 5)$ our equation will become:
\begin{align*}
\sum\limits_{k = 1}^m k! \text{ (mod 10)} & \equiv 1! + 2! + 3! + 4! +5!+ ...+  m! \text{ (mod 10)}\\
 & \equiv 1! + 2! + 3! + 4! + 0+ ...+  0 \text{ (mod 10)}\\
& \equiv 1! + 2! +3! +4! \text{ (mod 10)}
\end{align*}
Thus we have shown that if  $(m \geq 5)$, the last digit will be equal to the last digit when m = 4. Thus the only possible unique solutions can exists when ($m = 1$), ($m = 2$), ($m = 3$), ($m = 4$):
\[ (m = 1): \sum\limits_{k = 1}^1 k! \text{ (mod 10)} \equiv 1! \text{ (mod 10)} \equiv 1 \text{ (mod 10)}\]
\[ (m = 2): \sum\limits_{k = 1}^2 k! \text{ (mod 10)} \equiv 1! + 2! \text{ (mod 10)} \equiv  3 \text{ (mod 10)}\]
\[ (m = 3): \sum\limits_{k = 1}^3 k! \text{ (mod 10)} \equiv 1! + 2! + 3! \text{ (mod 10)} \equiv 9 \text{ (mod 10)}\]
\[ (m = 4): \sum\limits_{k = 1}^4 k! \text{ (mod 10)} \equiv 1! + 2! + 4! \text{ (mod 10)} \equiv 3 \text{ (mod 10)} \]
Thus we have shown that [2] will have unique solutions when m is 1,2,3 or 4. \\
Therefore we have 6 possible unique combinations which correspond to possible last digits:
\[ \text{(m = 1, where [1] is odd and [2] $<$ 5)}: 19^1 + \sum\limits_{k = 1}^1 k! \text{ (mod 10)} \equiv 9 + 1 \text{ (mod 10)} \equiv 0 \text{ (mod 10)} \]
\[ \text{(m = 2, where [1] is even and [2] $<$ 5)}: 19^2 + \sum\limits_{k = 1}^2 k! \text{ (mod 10)} \equiv 1 + 3 \text{ (mod 10)} \equiv 4 \text{ (mod 10)} \]
\[ \text{(m = 3, where [1] is odd and [2] $<$ 5)}: 19^3 + \sum\limits_{k = 1}^3 k! \text{ (mod 10)} \equiv 9 + 9 \text{ (mod 10)} \equiv 8 \text{ (mod 10)} \]
\[ \text{(m = 4, where [1] is even and [2] $<$ 5)}: 19^4 + \sum\limits_{k = 1}^4 k! \text{ (mod 10)} \equiv 1 + 3 \text{ (mod 10)} \equiv 4 \text{ (mod 10)} \]
\[ \text{(m = 5, where [1] is odd and $[2] \geq 5$)}: 19^5 + \sum\limits_{k = 1}^5 k! \text{ (mod 10)} \equiv 9 + 3 \text{ (mod 10)} \equiv 2 \text{ (mod 10)} \]
\[ \text{(m = 6, where [1] is even and $[2] \geq 5$)}: 19^6 + \sum\limits_{k = 1}^6 k! \text{ (mod 10)} \equiv 1 + 3 \text{ (mod 10)} \equiv 4 \text{ (mod 10)} \]
Since this exausts every possible combination for the natural $m$, only the elements [0,2,4,8] willl be contained in the set S:
\[ S = \left\{D\left(19^m + \sum\limits_{k = 1}^m k!\right) \colon m \in \mathbb N\right\} \]
\end{document}