\documentclass[11pt]{article}
\textwidth 15cm 
\textheight 21.3cm
\evensidemargin 6mm
\oddsidemargin 6mm
\topmargin -1.1cm
\setlength{\parskip}{1.5ex}


\usepackage{amsfonts,amsmath,amssymb,enumerate}



\begin{document}
\parindent=0pt

\textbf{Q05} We will start by letting $\epsilon_1 >0$ so that if $0 < |x+4| < \delta$, then $|f(x)-7|<\epsilon$.
\text{ASSUME WE CAN COMBINE 1 and 57....}

\[ f(x)^2 + f(x) + 1 -57 \implies \]
\[ f(x)^2 + f(x) -56 < \implies\]
\[ (f(x)+ 8)(f(x) - 7) < (f(x)+ 8) \epsilon\]


\text{ Crazy shity area....} Let $\epsilon_2 >0$ so that if $0 < |f(x)-7| < \delta_2$, then:
\[  f(x)^2 + f(x) + 1 -57  < \epsilon \]
We will let delta have a minimum value of 1 such that:
\[ 0 < |f(x)-7| < 1 \]
\[ 7 < f(x) < 8 \]
So thus we know that $7 < f(x) < 8$ are the only values we need to be concered with, thus we can find a fixed $\delta_2 \leq 1$:
\[ |f(x) + 8| < 8 + 8 = 16 \]
Return
\[ x^2 + x + 1 = x^2 + x -2 + 3 = (x + 2)(x-1) + 3 \]
\[ |(x+2)(x-1) + 3| \]
NTP: $x^2  + x + 1 = e$
\[ 0 <= |x+2| < delta \]

\[ 0 <= |x+2| < 1 \]
\[ -3 <= x < -1 \]
\[-3(x-1) + 3 = e\]
\[3(-(x-1) + 1) = e\]
\[3(x)\]


\end{document}