\documentclass[11pt]{article}
\textwidth 15cm 
\textheight 21.3cm
\evensidemargin 6mm
\oddsidemargin 6mm
\topmargin -1.1cm
\setlength{\parskip}{1.5ex}


\usepackage{amsfonts,amsmath,amssymb,enumerate}



\begin{document}
\parindent=0pt

\textbf{Q02A} If our we set [a] to be [7] our system of equations for $\mathbb Z_12$ and multiply the first equation by 5 and the second by 3 we get:
$$
\begin{cases}
[35][x] + [15][y] & \equiv [5]       (1)\\
[6][x] + [15][y] & \equiv [-3]      (2)
\end{cases}
$$
Subtracting equation (1) by equation (2) we will get the new equation: [29][y] = [8]. Note that [29] = [5] our equation will become:
\[ [5][x] \equiv [8] \]
We know that [5] and 12 are co-prime, and thus by INV (with integers) we know that [5] will have a mathimatical inverse, thus if we multiply both sides we will get:
\[ [5]^{-1}[5][y] \equiv [8][5]^{-1} \]
By definition $[5]^{-1}[5] = 1$ and we know that [5][5] = [1] (mod 12) and thus the mathimatical inverse of [5] is [5]. Our equation thus becomes:
\[ [x] \equiv [40] \equiv [4] \]
Plugging this into equation (2) we get that:
\begin{align*}
 [6][x] + [15][y] & \equiv [-3]  \\
 [15][y]  & \equiv [-3]  - [6][4] \\
 [3][y]  & \equiv [-3]  - [0] \\
 [3][y]  & \equiv [9]\\
\end{align*}
SFKMASKF
We know this condition will be satisfied if [y] = [3] and thus our solution set is given by: [x] = [4] and [y] = [3].


\textbf{Q02B} ...
\end{document}