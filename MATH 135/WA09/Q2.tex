\documentclass[11pt]{article}
\textwidth 15cm 
\textheight 21.3cm
\evensidemargin 6mm
\oddsidemargin 6mm
\topmargin -1.1cm
\setlength{\parskip}{1.5ex}


\usepackage{amsfonts,amsmath,amssymb,enumerate}



\begin{document}
\parindent=0pt

\textbf{Q02A} If our we set [a] to be [7] our system of equations for $\mathbb Z_{12}$ and multiply the first equation by 5 and the second by 3 we get:
$$
\begin{cases}
[35][x] + [15][y] & = [5] \\
[6][x] + [15][y] & = [-3] 
\end{cases}
$$
Subtracting the first equation by the second equation we will get the new equation: [29][y] = [8]. Note that [29] = [5] our equation will become:
\[ [5][x] = [8] \]
We know that [5] and 12 are co-prime, and thus by INV (with integers) we know that [5] will have a mathimatical inverse, thus if we multiply both sides we will get:
\[ [5]^{-1}[5][y] = [8][5]^{-1} \]
By definition $[5]^{-1}[5] = 1$ and we know that [5][5] = [1] (mod 12) and thus the mathimatical inverse of [5] is [5]. Our equation thus becomes:
\[ [x] = [40] = [4] \]
Plugging this into the second equation we get that:
\begin{align*}
 [6][x] + [15][y] & = [-3]  \\
 [15][y]  & = [-3]  - [6][4] \\
 [3][y]  & = [-3]  - [0] \\
 [3][y]  & = [9]
\end{align*}
We know this condition will be satisfied if [y] = [3] and thus our solution set is given by both: 
\[  \{ x \in \mathbb{Z} : x \equiv 4 \text{  (mod 12)} \} \]
\[  \{ y \in \mathbb{Z} : y \equiv 3 \text{  (mod 12)} \} \]

\textbf{Q02B} To start we will mutliply the first equation by 5 and the second by 3 so we will get:
$$
\begin{cases}
[5][a][x] + [15][y] & = [5]  \text{      (1)}\\
[6][x] + [15][y] & = [-3]    \text{      (2)}
\end{cases}
$$
Substracting the first equation by the second equation we will get the new equation:
\begin{align*}
 [5][a][x] - [6][x] & = [8] \\
 [x]([5][a] - [6]) & = [8] \\
\end{align*}
Note that we cancelled out the y's, this implies that number of possible solution set pairs is limited by the possible [x] solutions. Let [b] = [5][a] - [6]:
\[ [b][x] = [8] \]
MAT tells us a solution [x] will only exist if $d = \gcd(b,12)$ such that $d|8$. MAT also tells us that for a solution ($x_0$) the set of solutions is given by:
\[  [x_0], [x_0 + \frac{m}{d}], [x_0 + 2\frac{m}{d}]...[x_0 + (d-1)\frac{m}{d}]  \]
In other words this shows us that the number of possible solutions for [x] will be equal to d. Therefore all congruence classes that have 2 solutions will have $d = 2$, this can only happen in these cases:
$$
\begin{cases}
2 = \gcd(b, 12) = \gcd(2,12)\\
2 = \gcd(b, 12) = \gcd(10,12)\\
\end{cases}
$$
Thus we see that b has to equal either [2] or [10], solving for [2] we find that:
\begin{align*}
 [5][a] - [6]  & = [b] \\
          [5][a] & = [2]  - [6] \\
          [5][a]  & = [8] 
\end{align*}
We know from the previous question that the mathimatical inverse of [5] is [5] so if we multiply both sides by the inverse:
\begin{align*}
          [5]^{-1}[5][a]  & = [8][5]^{-1} \\
          [a]  & = [8][5] \\
          [a]  & = [4] 
\end{align*}
If we know solve for the second case, when b is [10] we get:
\begin{align*}
 [5][a] - [6]  & = [b] \\
          [5][a] & = [10]  - [6] \\
          [5][a]  & = [4]  \\
          [5]^{-1}[5][a]  & = [4][5]^{-1} \\
          [a]  & = [8][5] \\
          [a]  & = [8]
\end{align*}
Thus we know there are only 2 [x] solutions when [a] is [8] or [2], we also know there are only going to be 2 solution pairs ([x],[y]) when [a] is [8] or [2]

\textbf{Q02C} We know from the previous equations that the equation for a possible [x] value will be:
\[ [x]([5][a] - [6])  \equiv [8] \]
We also know that we want to find an [a] such that [x] has 6 solutions. MAT says this can only happen if
\[ d = \gcd(([5][a] - [6]), 12) = 6 \]
This is impossible as MAT tells us that a solution can only happen when $d\text{ }| \text{ } 8$ and since $6\not{|} \text{ } 8$, we know that no [a] exists such that there are 6 solution pairs ([x],[y]).


\end{document}