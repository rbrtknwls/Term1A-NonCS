\documentclass[11pt]{article}
\textwidth 15cm 
\textheight 21.3cm
\evensidemargin 6mm
\oddsidemargin 6mm
\topmargin -1.1cm
\setlength{\parskip}{1.5ex}


\usepackage{amsfonts,amsmath,amssymb,enumerate}



\begin{document}
\parindent=0pt

\textbf{Q02A} If our we set [a] to be [7] our system of equations and multiply the first equation by 5 and the second by 3 we get:
$$
\begin{cases}
[35][x] + [15][y] & = [5]       (1)\\
[6][x] + [15][y] & = [-3]      (2)
\end{cases}
$$
Subtracting equation (1) by equation (2) we will get the new equation: [29][y] = [8]. Note that [29] = [5] our equation will become:
\[ [5][x] = [8] \]
We know that [5] and 12 are co-prime, and thus by INV (with integers) we know that [5] will have a mathimatical inverse, thus if we multiply both sides we will get:
\[ [5]^{-1}[5][y] = [8][5]^{-1} \]
By definition $[5]^{-1}[5] = 1$ and we know that [5][5] = [1] (mod 12) and thus the mathimatical inverse of [5] is [5]. Our equation thus becomes:
\[ [x] = [40] = [4] \]
Plugging this into equation (2) we get that:
\begin{align*}
 [6][x] + [15][y] & = [-3]  \\
 [15][y]  & = [-3]  - [6][4] \\
 [3][y]  & = [-3]  - [0] \\
 [3][y]  & = [9]]\\
\end{align*}
We know this condition will be satisfied if [y] = [3] and thus our solution is: [x] = [4] and [y] = [3]


\textbf{Q01B} Visually we know that $\mathbb Z_7$ is a field, if we look at the multiplication table, each possible congruence class [a] has a corrisponding congruence class $[b]^{-1}$ such that:

\[ [a][b] = 1 \]

This happens because 7 is a prime and [a] is co-prime to 7. This means that $ d = \gcd([a],7]) = 1$ and by definition of MAT since $d|1$ there must be a solution [b] for each [a] that solves the above equality (which means [a] will have a multiplictive inverse).\\\\
On the other hand 8 is not prime and thus not all [a]'s are co-prime to 8. If [a] is not coprime to 7 this would result in $d = \gcd([a],8] \neq 1$ and thus MAT could not apply as $d\nmid1$, which means for all [b] of that a:

\[ [a][b] \neq 1 \]

Which means that [a] has no multiplicitive inverse. As an illistutive example lets consider [a] = [2] the multiplictive table will give us:

\begin{center}
 \begin{tabular}{||c | | c c c c c c c c||} 
 \hline
    $\cdot$ & [0] & [1] & [2] & [3] & [4] & [5] & [6] & [7]\\ 
 \hline
\hline
\text{[2]} & [0] & [2] & [4] & [6] & [0] & [2] & [4] & [6] \\ 
\hline
\end{tabular}
\end{center}

We can thus see that [1] is never a result and thus [a] will never have a multiplictive inverse.
\end{document}