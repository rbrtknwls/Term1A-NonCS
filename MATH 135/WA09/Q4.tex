\documentclass[11pt]{article}
\textwidth 15cm 
\textheight 21.3cm
\evensidemargin 6mm
\oddsidemargin 6mm
\topmargin -1.1cm
\setlength{\parskip}{1.5ex}


\usepackage{amsfonts,amsmath,amssymb,enumerate}



\begin{document}
\parindent=0pt

\textbf{Q04a} Let $n$ be a positive integer and let $a$ be an integer, to begin we will take the contrapositive of the starting statement, we will thus get that:
\[ \text{if a is odd} \implies (a^2)^{n-1} \equiv 1 \ \text{(mod $2^n$)} \]
We can prove this using induction;

Base Case: n = 1:
Subsituting $n$ for 1 into our starting equation we get:
\begin{align*}
 (a^2)^{n-1} & \equiv 1 \ \text{(mod $2^n$)} \\
 (a^2)^{1-1} & \equiv 1 \ \text{(mod $2^1$)} \\
 a^0 & \equiv 1 \ \text{(mod 2)} \\
 1 & \equiv 1 \ \text{(mod 2)} \\
\end{align*}
Thus proving the base case or when n is 1.

Inductive Hypothesis: Let a $k$ exist such that $k \geq 1$, we will also assume that any $(a^2)^{k-1} \equiv 1$ (mod $2^k)$. To start we will need to prove that $(a^2)^{k} \equiv 1$ (mod $2^{k+1})$.\\
\begin{align*}
 (a^2)^{k} & \equiv  (a^2)^{k-1}(a^2)\\
\end{align*}
Note from out hypothesis that:
\[ (a^2)^{k-1} \equiv 1 \ \ \text{(mod $2^k$)}  \]
Which is the same thing as for some integer $s$:
\[ (a^2)^{k-1} \equiv 1 +  s\cdot2^k\]
Thus plugging this in we get that:
\begin{align*}
 (a^2)^{k} & \equiv  (a^2)^{k-1}(a^2)\\
& \equiv  (1 +  s\cdot 2^k) (a^2)\\
\end{align*}
We know that since a is odd that means for some t,$ a^2$ can be expressed as:
\[ a^2 = 1 + t\cdot2 \]
Thus our equation becomes 
\begin{align*}
(1 +  s\cdot 2^k) (a^2) & \equiv (1 +  s\cdot 2^k) (1+t\cdot2)  \\
& \equiv  1 +  t\cdot 2 + s\cdot 2^k + s\cdot 2^k \cdot t \cdot2  \\
\end{align*}
Let c = b + t, we thus get:
\begin{align*}
1 +  t\cdot 2 + s\cdot 2^k + s\cdot 2^k \cdot t \cdot2  & \equiv   1 +   2\cdot c \cdot 2^k + s\cdot 2^k \cdot t \cdot2 \\
 & \equiv 1 +   c \cdot 2^{k+1} + s\cdot 2^{k+1} \cdot t 
\end{align*}
Thus applying mod we get that:
\begin{align*}
 (a^2)^{k} & \equiv 1 +   c \cdot 2^{k+1} + s\cdot 2^{k+1} \cdot t \\
& \equiv 1 +   c \cdot 2^{k+1} + s\cdot 2^{k+1} \cdot t \ \ \text{mod $2^{k+1}$} \\
& \equiv 1 +   c \cdot 0+ s\cdot0 \cdot t \ \ \text{mod $2^{k+1}$} \\
& \equiv 1 \ \ \text{mod $2^{k+1}$}
\end{align*}
Since the hypothesis is now correctly shown the induction is complete and thus we have proved the statement by contrapositive and induction/

\textbf{Q04b} First take the contrapositive of the statement, so we will get where p is prime:
\[ p \not{|} \ a \implies \ \text{if a is even or} (a^{2(p-1)})^n-1 \equiv 1 \ \ \text{(mod $2^np$)} \]
Lets look at the first case:
\[ p \not{|} \ a \implies (a^{2(p-1)})^n-1 \equiv 1 \ \ \text{(mod $2^np$)} \]
Since we are we are assuming the second case is false, this means that a is even. We know from the previous question that when a is odd it can be expressed as for some integer k:
\[ (a^2)^{k-1} \equiv 1 \ \text{(mod $2^k$)} \]
If we raise both sides to the power of (p-1) we end up getting:
\[ (a^2)^{(k-1)(p-1)} \equiv 1^(p-1) \ \text{(mod $2^k$)} \]
\[ (a^2)^{(k-1)(p-1)} \equiv 1 \ \text{(mod $2^k$)} \]
We can use FLT as p can not divide the left values and since $p|a$ and a is to the power of c:
\[ (a^2)^{(k-1)(p-1)} \equiv 1 \ \text{(mod $p$)} \]
We also know that the $\gcd(p, 2^n) = 1$ so we can use CRT:
\[ (a^2)^{(k-1)(p-1)} \equiv 1 \ \text{(mod $2^kp$)} \]
Thus proving the contrapositive as in the other case $p$ and never divide $a$ as $a$ is even and p by definition is an odd prime!
\end{document}