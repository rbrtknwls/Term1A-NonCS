
\documentclass[11pt]{article}
\textwidth 15cm 
\textheight 21.3cm
\evensidemargin 6mm
\oddsidemargin 6mm
\topmargin -1.1cm
\setlength{\parskip}{1.5ex}


\usepackage{amsfonts,amsmath,amssymb,enumerate}

\begin{document}

\textbf{Q02a} Let $n$ be a positive integer greater then 1, then by the Unique Factorization Theorem, $n$ can be expressed as:
\[  n = p_1^{\alpha_1}p_2^{\alpha_2}\cdots p_k^{\alpha_k} \]
where $p_1$,$p_2$,...,$p_k$, $k\geq1$ are a list distinct primes $\leq \sqrt{n}$ and $a_1$,$a_2$,...,$a_k$ are all non-negative integers. From the DFPF, we know that all of $n$'s divisors (known as the integer c) are of the form:
\[  c = p_1^{\beta_1}p_2^{\beta_2}\cdots p_k^{\beta_k} \text{, where } 0\leq\beta_i\leq\alpha_i \text{ for  }  i = 1,2,...,k\]
For any given prime ($p_i$), we will have $\alpha_i$ as 
\end{document}